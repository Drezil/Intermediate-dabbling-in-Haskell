\documentclass{beamer}

\usepackage[utf8]{inputenc}
\usepackage[T1]{fontenc}
\usepackage[ngerman]{babel}
\usepackage{graphicx} % Bilder
\usepackage{wrapfig} % Umflussbilder
\usepackage{multicol} % Multiple columns
\usepackage{minted} % Haskell source code
\usepackage{framed} % Frames around source code
\usepackage[framemethod=tikz]{mdframed} % Frames
\usepackage{verbatim} % \begin{comment}...\end{comment}
\usepackage{etoolbox} % manipulate minted
\AtBeginEnvironment{minted}{\fontsize{10}{10}\selectfont}
\AfterEndEnvironment{minted}{}

\mdfdefinestyle{fancy}{
  roundcorner=5pt,
  linewidth=4pt,
  linecolor=red!80,
  backgroundcolor=red!20
}
\newmdenv[style=fancy]{important}

% Stuff for Beamer
\beamertemplatenavigationsymbolsempty
\usetheme{Warsaw}

\title{Intermediate Dabbling in Haskell}

\begin{document}

%  \usebackgroundtemplate{\includegraphics[width=\paperwidth,height=\paperheight]{1.jpg}} 
  
%----------------------------------------------------------------------------------------  

  \begin{frame}
  \begin{center}
    \Huge\textbf{Intermediate Functional Programming in Haskell}\\ \bigskip
    \LARGE Universität Bielefeld, Sommersemester 2015\\ \bigskip
    \large Jonas Betzendahl \& Stefan Dresselhaus
    \end{center}
  \end{frame}

%----------------------------------------------------------------------------------------  
\begin{frame}[allowframebreaks]{Outline}
\frametitle{Übersicht}
\tableofcontents[hideallsubsections]
\end{frame}

\section{Parsing}

\subsection{Motivation}

\begin{frame}[fragile]
\large Wozu das ganze?\\
\normalsize In der \glqq echten Welt\grqq \ haben wir häufig Eingabedaten in verschiedensten Formaten:
\begin{itemize}
 \item Text (z.B. Config-Files, Log-Files)
 \item JSON (z.B. im Web-Kontext)
 \item XML (z.B. (X)HTML)
 \item Binärcode (z.B. 3D-Modelle, Netzwerkcode, ...)
\end{itemize}
\pause
Diese wollen wir nun in Haskell nutzbar machen, um sie aufzubereiten, zu filtern oder generell weiterzuverarbeiten.
\end{frame}

\begin{frame}[fragile]
\large Wie gehen wir das ganze an?\\
\normalsize Naiv: Textvergleiche, Patterns und reguläre Ausdrücke.\\
\bigskip
\pause
This is not the Haskell-way to do that.\\
\bigskip
\pause
In Haskell möchten wir gerne \textbf{kleine Teilprobleme lösen} (wie z.b. das Lesen eines Zeichens oder einer Zahl) und diese dann zu größeren Lösungen \textbf{kombinieren}.
\end{frame}

\subsection{Theorie}
\begin{frame}[fragile]
Kernstück für das stückweise Parsen bildet die Applicative-Typklasse mit ihrer monoidal-Erweiterung \glqq Alternative\grqq .\\
\pause
\bigskip
Was heisst das genau?\\
\bigskip
\pause
\glqq Alternative\grqq \  ist in der Lage viele Dinge zu probieren und dann das erste zurückzuliefern, was geklappt hat.\\
So ist man in der Lage verschiedene Möglichkeiten des weiter-parsens auszudrücken.
\end{frame}

\section{Arbeit am Beispiel}
\subsection{Beispiel}
\begin{frame}[fragile]
Kommen wir zunächst zu einem Beispiel. Gegeben ist folgendes Log, welches in Haskell übersetzt werden soll:
\small
\begin{verbatim}
2013-06-29 11:16:23 124.67.34.60 keyboard
2013-06-29 11:32:12 212.141.23.67 mouse
2013-06-29 11:33:08 212.141.23.67 monitor
2013-06-29 12:12:34 125.80.32.31 speakers
2013-06-29 12:51:50 101.40.50.62 keyboard
2013-06-29 13:10:45 103.29.60.13 mouse
\end{verbatim}
\normalsize
Wir haben hier ein Liste von Daten, IP-Addressen und Geräten, die irgendwie interagiert haben.
\end{frame}

\begin{frame}[fragile]
Zunächst schauen wir uns den Aufbau einer Zeile an
\begin{verbatim}
2013-06-29 11:16:23 124.67.34.60 keyboard
\end{verbatim}
\pause
Sie besteht aus
\begin{enumerate}
 \item einem Datum (YYYY-MM-DD hh:mm:ss)
 \item einer IP (0.0.0.0 - 255.255.255.255)
 \item einem Gerät (String als Identifier)
\end{enumerate}
\end{frame}

\subsection{Definition der Datenstrukturen}
\begin{frame}[fragile]
Für alles definieren wir nun unsere Wunsch-Datenstrukturen:\\\pause
\bigskip
Eine Zeile lässt sich darstellen als
\begin{minted}{haskell}
data LogZeile = LogZeile Datum IP Geraet
\end{minted}
und das gesamte Log als viele Zeilen:
\begin{minted}{haskell}
data Log = Log [LogZeile]
\end{minted}
\end{frame}

\begin{frame}[fragile]
Datum (YYYY-MM-DD hh:mm:ss) können wir darstellen als
\begin{minted}{haskell}
import Data.Time

data Datum = Datum
             { tag  :: Day
             , zeit :: TimeOfDay
             } deriving (Show, Eq)

> Datum (fromGregorian 2014 1 2) (TimeOfDay 13 37 0)
Datum {tag = 2014-01-02, zeit = 13:37:00}
\end{minted}

\end{frame}

\begin{frame}[fragile]
Eine IP (0.0.0.0 - 255.255.255.255) ist darstellbar als
\begin{minted}{haskell}
import Data.Word

data IP = IP Word8 Word8 Word8 Word8 deriving (Show,Eq)

> IP 13 37 13 37
IP 13 37 13 37
\end{minted}
mit Word8 als unsigned 8-Bit-Integer.\\
\pause
Wenn falsche Werte nicht darstellbar sind, dann haben wir sie auch nicht in unserem Programm als Problem.
\end{frame}

\begin{frame}[fragile]
Ein Gerät (String als Identifier) können wir nur als Solches definieren:
\begin{minted}{haskell}
data Geraet = Mouse 
            | Keyboard 
            | Monitor 
            | Speakers 
            deriving (Show,Eq)
\end{minted}
Hier können wir nachher bei Bedarf auch schnell welche hinzufügen und der Compiler meckert dann an allen stellen herum, wo wir nicht mehr alle Fälle abfangen.
\end{frame}

\subsection{Parsing in die Datenstrukturen}
\begin{frame}[fragile]
\begin{minted}{haskell}
> Datum (fromGregorian 2014 1 2) (TimeOfDay 13 37 0)
Datum {tag = 2014-01-02, zeit = 13:37:00}
\end{minted}
\pause
Mit attoparsec liest sich der Code fast von allein:
\small
\begin{minted}{haskell}
{-# LANGUAGE OverloadedStrings #-}
import Data.Time
import Data.Attoparsec.Char8

zeitParser :: Parser Datum
zeitParser = do
  y  <- count 4 digit; char '-'
  mm <- count 2 digit; char '-'
  d  <- count 2 digit; char ' '
  h  <- count 2 digit; char ':'
  m  <- count 2 digit; char ':'
  s  <- count 2 digit;
  return $
    Datum { tag  = fromGregorian (read y) (read mm) (read d)
          , zeit = TimeOfDay (read h) (read m) (read s)
          }
\end{minted}
\normalsize

\end{frame}

\begin{frame}[fragile]
Für die IP sieht der Code ähnlich aus:
\small
\begin{minted}{haskell}
{-# LANGUAGE OverloadedStrings #-}
import Data.Attoparsec.Char8
import Data.Word

parseIP :: Parser IP
parseIP = do
  d1 <- decimal
  char '.'
  d2 <- decimal
  char '.'
  d3 <- decimal
  char '.'
  d4 <- decimal
  return $ IP d1 d2 d3 d4
\end{minted}
\normalsize
\end{frame}

\begin{frame}[fragile]
Für das Gerät brauchen wir die Mächtigkeit von Alternativen:
\small
\begin{minted}{haskell}
{-# LANGUAGE OverloadedStrings #-}
import Data.Attoparsec.Char8
import Control.Applicative

geraetParser :: Parser Geraet
geraetParser =
     (string "mouse"    >> return Mouse)
 <|> (string "keyboard" >> return Keyboard)
 <|> (string "monitor"  >> return Monitor)
 <|> (string "speakers" >> return Speakers)
\end{minted}
\normalsize
\texttt{<|>} ist der alternativ-Operator, der das Rechte ausführt, wenn das Linke einen Fehler wirft.\\
\pause
Wir matchen hier jeweils auf den String, schmeissen das gematchte Ergebnis weg (den String selbst) und liefern unseren Datentyp zurück.
\end{frame}

\begin{frame}[fragile]
Für die gesamte Zeile packen wir einfach unsere Parser zusammen:
\small
\begin{minted}{haskell}
zeilenParser :: Parser LogZeile
zeilenParser = do
     datum <- zeitParser
     char ' '
     ip <- parseIP
     char ' '
     geraet <- geraetParser
     return $ LogZeile datum ip geraet
\end{minted}
\normalsize
\end{frame}

\begin{frame}[fragile]
Für das Log nehmen wir die many-Funktion aus \glqq Alternative\grqq :
\begin{minted}{haskell}
logParser :: Parser Log
logParser = many $ zeilenParser <* endOfLine
\end{minted}
\texttt{<*} (aus Applicative) fungiert hier als ein Parser-Kombinator, der erst links matched, dann rechts matched und dann das Ergebnis des Linken zurückliefert.\\
Hier verwenden wir diesen um das Zeilenende hinter jeder Zeile loszuwerden.
\end{frame}

\begin{frame}[fragile]
Nun schreiben wir ein kleines Testprogramm:
\begin{minted}{haskell}
main :: IO ()
main = do
     log <- B.readFile "log.txt"
     print $ parseOnly logParser log
\end{minted}
und führen es aus:
\tiny
\begin{verbatim}
Right [LogZeile (Datum {tag = 2013-06-29, zeit = 11:16:23}) (IP 124 67 34 60) Keyboard,
       LogZeile (Datum {tag = 2013-06-29, zeit = 11:32:12}) (IP 212 141 23 67) Mouse,
       LogZeile (Datum {tag = 2013-06-29, zeit = 11:33:08}) (IP 212 141 23 67) Monitor,
       LogZeile (Datum {tag = 2013-06-29, zeit = 12:12:34}) (IP 125 80 32 31) Speakers,
       LogZeile (Datum {tag = 2013-06-29, zeit = 12:51:50}) (IP 101 40 50 62) Keyboard,
       LogZeile (Datum {tag = 2013-06-29, zeit = 13:10:45}) (IP 103 29 60 13) Mouse]
\end{verbatim}
\normalsize

\end{frame}

\section{Parser-Funktionsweise}
\begin{frame}[fragile]
Wie funktioniert so ein Parser nun intern genau?\\
\bigskip
Wir werden im folgenden attoparsec nicht im Detail erklären, sondern eher die Idee, die dahinter steht.
\end{frame}

\subsection{Poor-mans parser}
\begin{frame}[fragile]
Eigentlich hätten wir gerne eine Funktion
\begin{minted}{haskell}
parse :: String -> a
\end{minted}
\pause
aber was machen wir, wenn etwas von dem String überbleibt, damit wir später weitere dinge parsen können?\\
\pause
Also brauchen wir
\begin{minted}{haskell}
parse :: String -> (a,String)
\end{minted}
\pause
Aber das Parsen kann auch Fehlschlagen. Und Fehlermeldungen wären auch schön.
\pause
\begin{minted}{haskell}
parse :: String -> (Either String a,String)
\end{minted}
oder alternativ
\begin{minted}{haskell}
parse :: String -> Either String (a,String)
\end{minted}
\end{frame}

\begin{frame}[fragile]
Im folgenden nehmen wir
\begin{minted}{haskell}
parse :: String -> (Either String a,String)
\end{minted}
als Parser an. So können wir z.b. weiterparsen, wenn ein Teil einen Fehler wirft. Außerdem wollen wir einen fancy type:
\begin{minted}{haskell}
data Parser a = 
     Parser {
            runParser :: String -> (Either String a, String)
     }
\end{minted}
\pause
Dies gibt uns (wie schon beim State) 2 Funktionen:
\begin{minted}{haskell}
Parser    :: (String -> (Either String a, String)) -> Parser a
runParser :: Parser a -> String -> (Either String a, String)
\end{minted}
\end{frame}

\begin{frame}[fragile]
Natürlich bildet unser Parser wieder eine Monade. So können wir verschiedene Parser miteinander zu großen kombinieren.\\
Um das besser zu verstehen, gehen wir im Folgenden die nötigen Definitionen (Functor, Applicative, Monad) durch.
\end{frame}

\subsection{F/A/M}
\begin{frame}[fragile]
\begin{minted}{haskell}
instance Functor Parser where
fmap f p = Parser $ \input ->
             let
               (res, rem) = runParser p input
             in
               (fmap f res, rem)
\end{minted}
\end{frame}

\begin{frame}[fragile]
\begin{minted}{haskell}
instance Applicative Parser where
  pure a    = Parser $ \input -> (Right a,input)
  pf <*> pa = Parser $ \input ->
                let
                  (rf, rem1) = runParser pf input
                  (ra, rem2) = runParser pa rem1
                in
                  (rf <*> ra, rem2)
\end{minted}
\end{frame}

\begin{frame}[fragile]
\begin{minted}{haskell}
instance Monad Parser where
  return a  = pure a
  pa >>= f  = Parser $ \input ->
                let
                  (ra, rem1) = runParser pa input
                in
                  case ra of
                    Right a  -> runParser (f a) rem1
                    Left err -> (Left err, rem1)
\end{minted}
\end{frame}

\begin{frame}[fragile]
Was fehlt uns noch?\\
\pause
\begin{itemize}
 \item Langweilige Definition simpler Parser (digit, lit, space, ...)
 \item Erweiterung mit Continuations (Text \glqq nachschieben\grqq)
 \item Implementation einer Standardfehlerbehandlung
\end{itemize}
\pause
Wir kümmern uns um das letzgenannte, indem wir Alternative implementieren
\end{frame}

\subsection{Alternative}
\begin{frame}[fragile]
\begin{minted}{haskell}
instance Alternative Parser where
  empty     = Parser $ \inp -> (Left "",inp)
  pa <|> pb = Parser $ \input -> 
                let
                  (ra, rem1) = runParser pa input
                in
                  case ra of
                    Right _ -> (ra, rem1)
                    _       -> runParser pb input
\end{minted}
Schlägt der Parser hier fehl, dann wird der bereits benutzte Input wiederhergestellt und ein weiterer Parser probiert.
\end{frame}

\begin{frame}[fragile]
Nun können wir ein paar Simple Dinge parsen:
\begin{minted}{haskell}
parse1 :: Parser Int
parse1 = Parser $ \inp -> case inp of
                    ('1':xs) -> (Right 1,xs)
                    x        -> (Left "no 1", x)

parse0 :: Parser Int
parse0 = Parser $ \inp -> case inp of
                    ('0':xs) -> (Right 0, xs)
                    x        -> (Left "no 0", x)
\end{minted}
\pause
Dies sind dann intrinsics oder interne Funktionen, die ohne einen Zugriff auf \glqq Parser\grqq nicht möglich sind. Somit können wir verhindern, dass Leute mit dem Eingabestring schindluder treiben.
\end{frame}

\begin{frame}[fragile]
Binärzahlen können wir dann wie folgt parsen:
\begin{minted}{haskell}
parseBin :: Parser Int
parseBin = parse0 <|> parse1

testParser :: Parser [Int]
testParser = many parseBin

main :: IO ()
main = print $ runParser testParser "101001"
-- (Right [1,0,1,0,0,1],"")
\end{minted}
\end{frame}





\end{document}