\documentclass[a4paper,10pt]{scrartcl}
\usepackage[utf8]{inputenc}
\usepackage[ngerman]{babel}
\usepackage{amsmath}
\usepackage{amsfonts}
\usepackage{amsthm}
\usepackage{geometry}
\usepackage{multicol}
\usepackage{minted}
\geometry{a4paper,left=30mm,right=30mm, top=3cm, bottom=2cm} 

\newcommand{\underfat}[1]{\underline{\textbf{#1}}}
\newcommand{\theuebungszettel}{4}

\parindent0pt

\begin{document}

\begin{center}
  \begin{huge}
    \underfat{Fortgeschrittene funktionale}\\
    \underfat{Programmierung in Haskell}\\
  \end{huge}
\begin{LARGE}
\textbf{Übungszettel \theuebungszettel}
\end{LARGE}
\end{center}
\section*{Aufgabe \theuebungszettel.1:}
In der Vorlesung haben wir besprochen, dass man eine Lens auch für nicht-existente Felder schreiben kann. Gegeben ein Zeitdatentyp
\begin{verbatim}
data TimeSplice = TS { days :: Int
                     , hours :: Int
                     , minutes :: Int
                     , seconds :: Float
                  }
                  deriving (Show, Eq)
\end{verbatim}
Schreiben sie einen Isomorphismus, welcher eine Operation auf Sekunden vollführt, aber weiterhin einen TimeSlice nimmt:
\begin{minted}{haskell}
convert :: Iso TimeSplice TimeSplice Float Float
convert = iso ...

alterSeconds :: (Float -> Float) -> TimeSplice -> TimeSplice
alterSeconds = over convert
\end{minted}

\end{document}