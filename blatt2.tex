\documentclass[a4paper,10pt]{scrartcl}
\usepackage[utf8]{inputenc}
\usepackage[ngerman]{babel}
\usepackage{amsmath}
\usepackage{amsfonts}
\usepackage{amsthm}
\usepackage{geometry}
\usepackage{multicol}
\usepackage{minted}
\geometry{a4paper,left=30mm,right=30mm, top=3cm, bottom=2cm} 

\newcommand{\underfat}[1]{\underline{\textbf{#1}}}
\newcommand{\theuebungszettel}{2}

\parindent0pt

\begin{document}

\begin{center}
  \begin{huge}
    \underfat{Fortgeschrittene funktionale}\\
    \underfat{Programmierung in Haskell}\\
  \end{huge}
\begin{LARGE}
\textbf{Übungszettel \theuebungszettel}
\end{LARGE}
\end{center}
\section*{Aufgabe \theuebungszettel.1:}
In der Vorlesung wurde neben dem Datentypen Maybe auch Either vorgestellt. Zur Erinnerung: Either ist definiert als
\begin{minted}{haskell}
data Either a b = Left a
                | Right b
\end{minted}
Erstellen sie hierzu die Instanzen für
\begin{itemize}
 \item Functor
       \begin{minted}{haskell}
  class Functor f where
      fmap :: (a -> b) -> f a -> f b
       \end{minted}
 \item Applicative
       \begin{minted}{haskell}
  class Functor f => Applicative f where
      pure  :: a -> f a
      (<*>) :: f (a -> b) -> f a -> f b
       \end{minted}
 \item Monad
       \begin{minted}{haskell}
  class Applicative m => Monad m where
      return :: a -> m a
      (>>=)  :: m a -> (a -> m b) -> m b
       \end{minted}
\end{itemize}
\section*{Aufgabe \theuebungszettel.2:}
Ein weiterer einfacher Datentyp ist \texttt{Identity}, welcher keinen Effekt hat. Definiert ist \texttt{Identity} als
\begin{minted}{haskell}
newtype Identity a = {runIdentity :: a}
\end{minted}
Dieses definiert 2 Funktionen: Eine, um eine \texttt{Identity} zu kreieren und eine um wieder an ihren Inhalt zu kommen:
\begin{minted}{haskell}
Identity    :: a -> Identity a
runIdentity :: Identity a -> a
\end{minted}
Erstellen sie hier ebenfalls Instanzen für Functor, Applicative und Monad \textbf{ohne} durch Patternmatching an den Inhalt zu kommen. (Nutzen sie hierfür die Funktion \texttt{runIdentity}).

\section*{Aufgabe \theuebungszettel.2:}
In der Vorlesung wurde ebenfalls die State-Monade besprochen. Die Definition ist wie folgt:
\begin{minted}{haskell}
newtype State s a = {runState :: s -> (a,s)}
\end{minted}
Dieses definiert ebenfalls 2 Funktionen:
\begin{minted}{haskell}
State    :: (s -> (a,s)) -> State s a
runState :: State s a -> s -> (a,s)
\end{minted}
Erstellen sie hier ebenfalls die Instanzen:
\begin{minted}{haskell}
Functor (State s)
Applicative (State s)
Monad (State s)
\end{minted}

\end{document}