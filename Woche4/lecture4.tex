\documentclass{beamer}

\usepackage[utf8]{inputenc}
\usepackage[T1]{fontenc}
\usepackage[ngerman]{babel}
\usepackage{graphicx} % Bilder
\usepackage{wrapfig} % Umflussbilder
\usepackage{multicol} % Multiple columns
\usepackage{minted} % Haskell source code
\usepackage{framed} % Frames around source code
\usepackage[framemethod=tikz]{mdframed} % Frames
\usepackage{verbatim} % \begin{comment}...\end{comment}
\usepackage{etoolbox} % manipulate minted
\AtBeginEnvironment{minted}{\fontsize{10}{10}\selectfont}
\AfterEndEnvironment{minted}{}

\mdfdefinestyle{fancy}{
  roundcorner=5pt,
  linewidth=4pt,
  linecolor=red!80,
  backgroundcolor=red!20
}
\newmdenv[style=fancy]{important}

% redifine \em for \emph to use bold instead of italics
\makeatletter
\DeclareRobustCommand{\em}{%
  \@nomath\em \if b\expandafter\@car\f@series\@nil
  \normalfont \else \bfseries \fi}
\makeatother

% Stuff for Beamer
\beamertemplatenavigationsymbolsempty
\usetheme{Warsaw}

\title{Fortgeschrittene Funktionale Programmierung in Haskell}

\begin{document}

%  \usebackgroundtemplate{\includegraphics[width=\paperwidth,height=\paperheight]{1.jpg}} 
  
%----------------------------------------------------------------------------------------  

  \begin{frame}
  \begin{center}
    \huge\textbf{Fortgeschrittene Funktionale Programmierung in Haskell}\\ \bigskip
    \LARGE Universität Bielefeld, Sommersemester 2015\\ \bigskip
    \large Jonas Betzendahl \& Stefan Dresselhaus
    \end{center}
  \end{frame}

%----------------------------------------------------------------------------------------  
\begin{frame}[allowframebreaks]{Outline}
\frametitle{Übersicht}
\tableofcontents[hideallsubsections]
\end{frame}

\section{Wiederholung und Beispiele}

\subsection{Definitionen}

\begin{frame}[fragile]
Die zentrale Struktur mit der wir uns im Folgenden beschäftigen ist
\pause
\begin{minted}{haskell}
data V3 a = V3 a a a
\end{minted}
ein 3-dimensionaler Vektor.\\ \par
\pause
Wieso?
\pause
\begin{itemize}
 \item Sehr viele Algorithmen in der Informatik basieren auf vektoriellen Daten
 \pause
 \item Die Operationen für den 3D-Fall unterscheiden sich nur unwesentlich vom n-D-Fall
 \pause
 \item Die Konzepte sollten alle bereits bekannt sein, daher geht es nur um die Umsetzung
\end{itemize}
\end{frame}

\begin{frame}[fragile]
Was für Operationen soll so ein Vektor unterstützen?
\pause
\begin{overprint}
\onslide<2>
\begin{itemize}
 \item Skalare Multiplikation
\end{itemize}
\onslide<3>
\begin{itemize}
 \item Skalare Multiplikation
 \item Vektoraddition
\end{itemize}
\onslide<4>
\begin{itemize}
 \item Skalare Multiplikation
 \item Vektoraddition
 \item Skalarprodukt
\end{itemize}
\onslide<5>
\begin{itemize}
 \item Skalare Multiplikation
 \item Vektoraddition
 \item Skalarprodukt
\end{itemize}
\begin{minted}{haskell}
mul :: Num a => a -> V3 a -> V3 a
\end{minted}
\onslide<6>
\begin{itemize}
 \item Anwendung von (*x) auf jede Komponente
 \item Vektoraddition
 \item Skalarprodukt
\end{itemize}
\begin{minted}{haskell}
mul :: Num a => a -> V3 a -> V3 a
(*2) :: Num a => a -> a
\end{minted}
\onslide<7>
\begin{itemize}
 \item Anwendung derselben Funktion auf jede Komponente
 \item Vektoraddition
 \item Skalarprodukt
\end{itemize}
\begin{minted}{haskell}
vmap :: (a -> a) -> V3 a -> V3 a
\end{minted}
\onslide<8>
\begin{itemize}
 \item Anwendung derselben Funktion auf jede Komponente
 \item Vektoraddition
 \item Skalarprodukt
\end{itemize}
\begin{minted}{haskell}
vmap :: (a -> b) -> V3 a -> V3 b
\end{minted}
\onslide<9>
\begin{itemize}
 \item Anwendung derselben Funktion auf jede Komponente
 \item Vektoraddition
 \item Skalarprodukt
\end{itemize}
\begin{minted}{haskell}
vmap :: (a -> b) -> V3 a -> V3 b
vadd :: Num a => V3 a -> V3 a -> V3 a
\end{minted}
\onslide<10>
\begin{itemize}
 \item Anwendung derselben Funktion auf jede Komponente
 \item Vektoraddition
 \item Skalarprodukt
\end{itemize}
\begin{minted}{haskell}
vmap :: (a -> b) -> V3 a -> V3 b
vadd :: Num a => V3 a -> V3 a -> V3 a
(+) :: Num a => a -> a -> a
\end{minted}
\onslide<11>
\begin{itemize}
 \item Anwendung derselben Funktion auf jede Komponente
 \item Anwendung einer Funktion mit 2 Argumenten auf 2 Vektoren
 \item Skalarprodukt
\end{itemize}
\begin{minted}{haskell}
vmap :: (a -> b) -> V3 a -> V3 b
vapply :: (a -> a -> a) -> V3 a -> V3 a -> V3 a
(+) :: Num a => a -> a -> a
\end{minted}
\onslide<12>
\begin{itemize}
 \item Anwendung derselben Funktion auf jede Komponente
 \item Anwendung einer Funktion mit 2 Argumenten auf 2 Vektoren
 \item Skalarprodukt
\end{itemize}
\begin{minted}{haskell}
vmap :: (a -> b) -> V3 a -> V3 b
vapply :: (a -> b -> c) -> V3 a -> V3 b -> V3 c
(+) :: Num a => a -> a -> a
\end{minted}
\onslide<13>
\begin{itemize}
 \item Anwendung derselben Funktion auf jede Komponente
 \item Anwendung einer Funktion mit 2 Argumenten auf 2 Vektoren
 \item Skalarprodukt
\end{itemize}
\begin{minted}{haskell}
vmap :: (a -> b) -> V3 a -> V3 b
vapply :: (a -> b -> c) -> V3 a -> V3 b -> V3 c
vdot :: Num a => V3 a -> V3 a -> a
\end{minted}
\onslide<14>
\begin{itemize}
 \item Anwendung derselben Funktion auf jede Komponente
 \item Anwendung einer Funktion mit 2 Argumenten auf 2 Vektoren
 \item Erst \texttt{vapply (*)}, dann Zusammenfassen mit \texttt{+}
\end{itemize}
\begin{minted}{haskell}
vmap :: (a -> b) -> V3 a -> V3 b
vapply :: (a -> b -> c) -> V3 a -> V3 b -> V3 c
vdot :: Num a => V3 a -> V3 a -> a
vcompress :: V3 a -> a
\end{minted}
\onslide<15>
\begin{itemize}
 \item Anwendung derselben Funktion auf jede Komponente
 \item Anwendung einer Funktion mit 2 Argumenten auf 2 Vektoren
 \item Erst \texttt{vapply (*)}, dann Zusammenfassen mit \texttt{+}
\end{itemize}
\begin{minted}{haskell}
vmap :: (a -> b) -> V3 a -> V3 b
vapply :: (a -> b -> c) -> V3 a -> V3 b -> V3 c
vdot :: Num a => V3 a -> V3 a -> a
vfold :: (a -> a -> a) -> V3 a -> a
\end{minted}
\end{overprint}
\end{frame}

\subsection{Implementation}

\begin{frame}[fragile]
\begin{overprint}
\onslide<1>
\begin{minted}{haskell}
vmap :: (a -> b) -> V3 a -> V3 b
\end{minted}
\onslide<2>
\begin{minted}{haskell}
vmap :: (a -> b) -> V3 a -> V3 b
vmap f (V3 x y z) = ?
\end{minted}
\onslide<3->
\begin{minted}{haskell}
vmap :: (a -> b) -> V3 a -> V3 b
vmap f (V3 x y z) = V3 (f x) (f y) (f z)
\end{minted}
\end{overprint}
\pause
\pause
\pause
Dieses Muster haben wir schon häufiger gesehen.\\\par
\pause
Es ist ein \texttt{Functor}.
\begin{minted}{haskell}
class Functor f where
  fmap :: (a -> b) -> f a -> f b
\end{minted}
\pause
\begin{minted}{haskell}
instance Functor V3 where
  fmap :: (a -> b) -> V3 a -> V3 b
  fmap = vmap
\end{minted}

\end{frame}

\begin{frame}[fragile]
\begin{overprint}
\onslide<1>
\begin{minted}{haskell}
vapply :: (a -> b -> c) -> V3 a -> V3 b -> V3 c
\end{minted}
\onslide<2>
\begin{minted}{haskell}
vapply :: (a -> b -> c) -> V3 a -> V3 b -> V3 c
vapply f (V3 x y z) (V3 a b c) = ?
\end{minted}
\onslide<3->
\begin{minted}{haskell}
vapply :: (a -> b -> c) -> V3 a -> V3 b -> V3 c
vapply f (V3 x y z) (V3 a b c) = V3 (f x a) (f y b) (f z c)
\end{minted}
\end{overprint}
\pause
\pause
\pause
Wenn wir schon generisch sind: Was machen wir bei einer Funktion mit 3 Argumenten? \pause
 Was bei 4 Argumenten? \pause Was bei n Argumenten?\pause \\\bigskip
Funktionen sind gecurried. Wir brauchen immer nur \glqq ein Argument mehr\grqq \ verarbeiten.
\end{frame}

\begin{frame}[fragile]
Gehen wir das mal an einem Beispiel durch:
\begin{overprint}
\onslide<1>
\begin{minted}{haskell}
vapply (*) (V3 1 2 3)          (V3 4 5 6)
\end{minted}
\onslide<2>
\begin{minted}{haskell}
vapply (*) (V3 1 2 3)          (V3 4 5 6)
           (V3 (1*) (2*) (3*)) (V3 4 5 6)
\end{minted}
\onslide<3>
\begin{minted}{haskell}
vapply (*) (V3 1 2 3)          (V3 4 5 6)
           (V3 (1*) (2*) (3*)) (V3 4 5 6)
           (V3 (1*4) (2*5) (3*6))
\end{minted}
\onslide<4->
\begin{minted}{haskell}
vapply (*) (V3 1 2 3)          (V3 4 5 6)
           (V3 (1*) (2*) (3*)) (V3 4 5 6)
           (V3 (1*4) (2*5) (3*6))
           (V3   4    10    18)
\end{minted}
\end{overprint}
\pause
\pause
\pause
\pause
Wie können wir nun dieses \texttt{V3 (1*) (2*) (3*)} erzeugen?
\pause
\begin{overprint}
\onslide<6>
\begin{minted}{haskell}
(*) :: Num a => a -> a -> a
\end{minted}
\onslide<7>
\begin{minted}{haskell}
(*) :: Num a => a -> (a -> a)
\end{minted}
\onslide<8>
\begin{minted}{haskell}
(*) :: Num a => a -> b
\end{minted}
\texttt{b = (a -> a)}
\onslide<9>
\begin{minted}{haskell}
(*) :: Num a => a -> b
app :: (a ->  a -> a ) -> V3 a -> V3 (a -> a)
\end{minted}
\texttt{b = (a -> a)}
\onslide<10>
\begin{minted}{haskell}
(*) :: Num a => a -> b
app :: (a -> (a -> a)) -> V3 a -> V3 (a -> a)
\end{minted}
\texttt{b = (a -> a)}
\onslide<11>
\begin{minted}{haskell}
(*) :: Num a => a -> b
app :: (a -> b) -> V3 a -> V3 b
\end{minted}
\texttt{b = (a -> a)}
\onslide<12>
\begin{minted}{haskell}
(*) :: Num a => a -> b
fmap :: (a -> b) -> V3 a -> V3 b
\end{minted}
\texttt{b = (a -> a)}
\end{overprint}


\end{frame}

\begin{frame}[fragile]
\begin{overprint}
\onslide<1>
\begin{minted}{haskell}
fmap (*) (V3 1 2 3) = V3 (1*) (2*) (3*)
\end{minted}
\onslide<2>
\begin{minted}{haskell}
V3 (1*) (2*) (3*) :: V3 (a -> a)
\end{minted}
\onslide<3>
\begin{minted}{haskell}
V3 (1*) (2*) (3*) :: V3 (a -> a)
V3 4 5 6          :: V3 a
\end{minted}
\onslide<4>
\begin{minted}{haskell}
V3 (1*) (2*) (3*) :: V3 (a -> a)
V3 4 5 6          :: V3 a
apply :: V3 (a -> a) -> V3 a -> V3 a
\end{minted}
\onslide<5->
\begin{minted}{haskell}
V3 (1*) (2*) (3*) :: V3 (a -> a)
V3 4 5 6          :: V3 a
apply :: V3 (a -> b) -> V3 a -> V3 b
\end{minted}
\end{overprint}
\pause
\pause
\pause
\pause
\pause
Dies sollte uns auch bekannt vorkommen. \pause Es ist ein \texttt{Applicative}.\pause \\\bigskip
Applicative greift in unserem Beispiel auf \texttt{fmap} zurück um die Funktion in den Kontext zu bekommen und sofort anzuwenden.\\\pause
Aber eigentlich könnten wir auch mit
\begin{minted}{haskell}
V3 ((*) (*) (*)) :: V3 (a -> a -> a)
\end{minted}
anfangen und 2x \texttt{apply} aufrufen.\\\par
\pause
Hierfür brauchen wir noch eine Funktion
\begin{overprint}
\onslide<10>
\begin{minted}{haskell}
ins :: (a -> a -> a) -> V3 (a -> a -> a)
\end{minted}
\onslide<11>
\begin{minted}{haskell}
ins :: b             -> V3 b
\end{minted}
\texttt{b = (a -> a -> a)}
\end{overprint}
\end{frame}

\begin{frame}[fragile]
Somit haben wir alles für unser \texttt{Applicative} beisammen:
\begin{minted}{haskell}
class Applicative f where
  pure :: a -> f a
  <*> :: f (a -> b) -> f a -> f b
\end{minted}
\pause
Die Instanz für unser \texttt{V3} ist auch schnell geschrieben:
\begin{overprint}
\onslide<2>
\begin{minted}{haskell}
instance Applicative V3 where
  pure f = undefined
  vf <*> vx = undefined
\end{minted}
\onslide<3>
\begin{minted}{haskell}
instance Applicative V3 where
  pure f = V3 f f f
  vf <*> vx = undefined
\end{minted}
\onslide<4>
\begin{minted}{haskell}
instance Applicative V3 where
  pure f = V3 f f f
  (V3 f g h) <*> (V3 x y z) = undefined
\end{minted}
\onslide<5->
\begin{minted}{haskell}
instance Applicative V3 where
  pure f = V3 f f f
  (V3 f g h) <*> (V3 x y z) = V3 (f x) (g y) (h z)
\end{minted}
\end{overprint}
\pause
\pause
\pause
\pause
Somit fällt unsere Vektoraddition zusammen auf:
\begin{overprint}
\onslide<6>
\begin{minted}{haskell}
vadd x y =  pure (+) <*> x  <*> y
\end{minted}
\onslide<7>
\begin{minted}{haskell}
vadd x y = (fmap (+)     x) <*> y
\end{minted}
\onslide<8>
\begin{minted}{haskell}
vadd x y =       (+) <$> x  <*> y
\end{minted}
\end{overprint}
\end{frame}


\begin{frame}[fragile]
Wozu braucht man Applicative noch?\\\par \pause
Nehmen wir an wir hätten
\begin{minted}{haskell}
data Person = Person { name :: String
                     , age :: Int
                     , zip :: Int
                     , gen :: Gender
                     }
\end{minted}
\pause
und
\begin{minted}{haskell}
let names = V3 "Alice" "Bob" "Charlie"
let ages = V3 20 30 40
let zips = V3 33333 44444 55555
let gens = V3 Female Male Undecided
\end{minted}
\end{frame}

\begin{frame}[fragile]
dann können wir einfach
\begin{minted}{haskell}
let all = Person <$> names
                 <*> ages
                 <*> zips
                 <*> gens
\end{minted}
machen \pause und erhalten:
\begin{minted}{haskell}
all = V3 (Person "Alice"   20 33333 Female)
         (Person "Bob"     30 44444 Male)
         (Person "Charlie" 40 55555 Undecided)
\end{minted}

\end{frame}


\begin{frame}[fragile]
Dieser Fall ist für den \texttt{V3} etwas konstruiert.\\\par
\pause
Ein Beispiel aus einer echten Applikation wäre:
\begin{minted}{haskell}
personForm :: Html -> MForm Handler (FormResult Person, Widget)
personForm = renderDivs $ Person
    <$> areq textField "Name" Nothing
    <*> areq (jqueryDayField def
        { jdsChangeYear = True -- give a year dropdown
        , jdsYearRange = "1900:-5" -- 1900 till five years ago
        }) "Birthday" Nothing
    <*> aopt textField "Favorite color" Nothing
    <*> areq emailField "Email address" Nothing
    <*> aopt urlField "Website" Nothing
\end{minted}
\end{frame}

\begin{frame}[fragile]
Kommen wir aber nun zurück zu unserem \texttt{V3}:\\
Wir möchten noch ein Skalarprodukt definieren. Die Hälfte haben wir schon:
\begin{overprint}
\onslide<1-4>
\begin{minted}{haskell}
vdot :: V3 a -> V3 a -> a
vdot x y = _ ((*) <$> x <*> y)
\end{minted}
\onslide<5->
\begin{minted}{haskell}
vdot :: V3 a -> V3 a -> a
vdot x y = compress (+) ((*) <$> x <*> y)
\end{minted}
\end{overprint}
\pause
Wir brauchen noch eine Funktion
\begin{overprint}
\onslide<2>
\begin{minted}{haskell}
compress :: (a -> a -> a) -> V3 a -> a
\end{minted}
\onslide<3>
\begin{minted}{haskell}
compress :: (a -> a -> a) -> V3 a -> a
compress f (V3 x y z) = ?
\end{minted}
\onslide<4->
\begin{minted}{haskell}
compress :: (a -> a -> a) -> V3 a -> a
compress f (V3 x y z) = f (f x y) z
\end{minted}
\end{overprint}
\pause
\pause
\pause
\pause
Dies ist ein Spezialfall des generischen Faltens von Datestrukturen.\\\par
\pause
Man kann sich einen Vektor auch als Liste mit 3 Elementen vorstellen und auf Listen kennen wir bereits \texttt{foldl}, \texttt{foldr}, \dots
\\\par \pause
Also wäre es besser, wenn unser \texttt{V3} in dieser generischen Klasse sitzt, als wenn jeder Nutzer nachsehen muss, was nun \texttt{compress} genau tut.
\end{frame}

\begin{frame}[fragile]
Was heisst dieses \texttt{foldr} nochmal?\\\par
\pause
Sehen wir uns ein Beispiel an:
\begin{overprint}
\onslide<2>
\begin{minted}{haskell}
foldr (+) 0     [1,2,3,4]
\end{minted}
\onslide<3>
\begin{minted}{haskell}
foldr (+) (4+0) [1,2,3]
\end{minted}
\onslide<4>
\begin{minted}{haskell}
foldr (+) 4     [1,2,3]
\end{minted}
\onslide<5>
\begin{minted}{haskell}
foldr (+) (4+3) [1,2]
\end{minted}
\onslide<6>
\begin{minted}{haskell}
foldr (+) 7     [1,2]
\end{minted}
\onslide<7>
\begin{minted}{haskell}
foldr (+) (7+2) [1]
\end{minted}
\onslide<8>
\begin{minted}{haskell}
foldr (+) 9     [1]
\end{minted}
\onslide<9>
\begin{minted}{haskell}
foldr (+) (9+1) []
\end{minted}
\onslide<10->
\begin{minted}{haskell}
foldr (+) 10    []
=> 10
\end{minted}
\end{overprint}

\end{frame}


\begin{frame}[fragile]
Die Klasse hierzu heisst \texttt{Foldable}:
\begin{minted}{haskell}
class Foldable f where
  foldr :: (a -> b -> b) -> b -> f a -> b
oder
  foldMap :: Monoid m => (a -> m) -> f a -> m
\end{minted}
\pause
Beide Definitionen sind gleichwertig.
\pause
\begin{minted}{haskell}
instance Foldable V3 where
  foldr f s (V3 x y z) = f x $ f y $ f z s
\end{minted}
\pause
oder
\begin{minted}{haskell}
instance Foldable V3 where
  foldMap f (V3 x y z) = f x `mappend` f y `mappend` f z
\end{minted}
\pause
Diese Klasse definiert dann zahlreiche weitere Funktionen. Unter anderem auch
\begin{minted}{haskell}
foldr1 :: (a -> a -> a) -> f a -> a
\end{minted}
welches einfach nur \texttt{foldr} mit dem ersten Element als Startwert ist.
\end{frame}

\begin{frame}[fragile]
Wozu brauchen wir diese Abstraktion?\\\pause
Zunächst haben wir für unser Skalarprodukt
\begin{minted}{haskell}
vdot x y = foldr1 (+) ((*) <$> x <*> y)
\end{minted}
\pause
Aber wenn wir jetzt z.B. die Norm berechnen wollen, müssen wir die Wurzel ziehen. \pause Diese ist z.B. auf \texttt{Int} nicht definiert. \pause Somit können wir \emph{nicht} schreiben:
\begin{minted}{haskell}
vnorm x y = sqrt $ foldr1 (+) ((*) <$> x <*> y)
\end{minted}
\pause
Wir müssen also irgendwo die Umwandlung von \texttt{Int -> Double} machen. \\\par 
\pause
Dazu einige Beispiele:
\begin{minted}{haskell}
vnorm x y = sqrt $ fromIntegral $ foldr1 (+) ((*) <$> x <*> y)
vnorm x y = sqrt $ foldr ((+) . fromIntegral) 0
                                             ((*) <$> x <*> y)
vnorm x y = sqrt $ foldr1 (+) ((*) <$> (fromIntegral <$> x)
                                   <*> (fromIntegral <$> y))
\end{minted}
\pause
Diese Funktionen unterscheiden sich in der Laufzeit - und je nach Datentyp auch im Ergebnis.
\end{frame}

\subsection{Zusammenfassung}
\begin{frame}[fragile]
\begin{minted}{haskell}
module V3 where
import Control.Applicative
import Data.Foldable
import Data.Monoid
import Prelude hiding (foldr1)

data V3 a = V3 a a a
instance Functor V3 where
  fmap f (V3 x y z) = V3 (f x) (f y) (f z)
instance Applicative V3 where
  pure f = V3 f f f
  (V3 f g h) <*> (V3 x y z) = V3 (f x) (g y) (h z)
instance Foldable V3 where
  foldMap f (V3 x y z) = f x `mappend` f y `mappend` f z

vmul s x = (s*) <$> x
vadd x y = (+) <$> x <*> y
vdot x y = foldr1 (+) $ (*) <$> x <*> y
vnorm x y = sqrt $ vdot x y
vnorm' x y = sqrt . fromIntegral $ vdot x y
\end{minted}
\end{frame}

\begin{frame}[fragile]
Interessant wird es, wenn wir uns die Typen von unseren Funktionen geben lassen:
\begin{minted}{haskell}
vmul  :: (Num b, Functor f)
         => b -> f b -> f b
vadd  :: (Applicative f, Num b)
         => f b -> f b -> f b
vdot  :: (Foldable t, Applicative t, Num a)
         => t a -> t a -> a
vnorm :: (Foldable t, Applicative t, Floating s)
         => t s -> t s -> s
vnorm':: (Foldable t, Applicative t, Integral s, Floating c)
         => t s -> t s -> c
\end{minted}
\pause
Hier ist gar nicht mehr die Rede von \texttt{V3}.\\\par
\pause
Somit können wir dieselben Funktionen auch für \texttt{V2}, \texttt{V4} etc. benutzen, wenn wir \texttt{Functor}, \texttt{Applicative} und \texttt{Foldable} definieren.\\\par
\pause
Damit läuft das \glqq programmieren\grqq \ nicht mehr auf das erneute Implementieren, sondern nur auf das Instanzieren hinaus.
\end{frame}


\section{Parsing}

\subsection{Motivation}

\begin{frame}[fragile]
\large Wozu das ganze?\\
\normalsize In der \glqq echten Welt\grqq \ haben wir häufig Eingabedaten in verschiedensten Formaten:
\begin{itemize}
 \item Text (z.B. Config-Files, Log-Files)
 \item JSON (z.B. im Web-Kontext)
 \item XML (z.B. (X)HTML)
 \item Binärcode (z.B. 3D-Modelle, Netzwerkcode, ...)
\end{itemize}
\pause
Diese wollen wir nun in Haskell nutzbar machen, um sie aufzubereiten, zu filtern oder generell weiterzuverarbeiten.
\end{frame}

\begin{frame}[fragile]
\large Wie gehen wir das ganze an?\\
\normalsize Naiv: Textvergleiche, Patterns und reguläre Ausdrücke.\\
\bigskip
\pause
This is not the Haskell-way to do that.\\
\bigskip
\pause
In Haskell möchten wir gerne \textbf{kleine Teilprobleme lösen} (wie z.b. das Lesen eines Zeichens oder einer Zahl) und diese dann zu größeren Lösungen \textbf{kombinieren}.
\end{frame}

\subsection{Theorie}
\begin{frame}[fragile]
Kernstück für das stückweise Parsen bildet die Applicative-Typklasse mit ihrer Erweiterung \glqq Alternative\grqq .\\
\pause
\bigskip
Was heisst das genau?\\
\bigskip
\pause
\glqq Alternative\grqq \  ist in der Lage viele Dinge zu probieren und dann das erste zurückzuliefern, was geklappt hat.\\
So ist man in der Lage verschiedene Möglichkeiten des weiter-parsens auszudrücken.
\end{frame}

\section{Arbeit am Beispiel}
\subsection{Beispiel}
\begin{frame}[fragile]
Kommen wir zunächst zu einem Beispiel. Gegeben ist folgendes Log, welches in Haskell übersetzt werden soll:
\small
\begin{verbatim}
2013-06-29 11:16:23 124.67.34.60 keyboard
2013-06-29 11:32:12 212.141.23.67 mouse
2013-06-29 11:33:08 212.141.23.67 monitor
2013-06-29 12:12:34 125.80.32.31 speakers
2013-06-29 12:51:50 101.40.50.62 keyboard
2013-06-29 13:10:45 103.29.60.13 mouse
\end{verbatim}
\normalsize
Wir haben hier ein Liste von Daten, IP-Addressen und Geräten, die irgendwie interagiert haben.
\end{frame}

\begin{frame}[fragile]
Zunächst schauen wir uns den Aufbau einer Zeile an
\begin{verbatim}
2013-06-29 11:16:23 124.67.34.60 keyboard
\end{verbatim}
\pause
Sie besteht aus
\begin{enumerate}
 \item einem Datum (YYYY-MM-DD hh:mm:ss)
 \item einer IP (0.0.0.0 - 255.255.255.255)
 \item einem Gerät (String als Identifier)
\end{enumerate}
\end{frame}

\subsection{Definition der Datenstrukturen}
\begin{frame}[fragile]
Für alles definieren wir nun unsere Wunsch-Datenstrukturen:\\\pause
\bigskip
Eine Zeile lässt sich darstellen als
\begin{minted}{haskell}
data LogZeile = LogZeile Datum IP Geraet
\end{minted}
und das gesamte Log als viele Zeilen:
\begin{minted}{haskell}
data Log = Log [LogZeile]
\end{minted}
\end{frame}

\begin{frame}[fragile]
Datum (YYYY-MM-DD hh:mm:ss) können wir darstellen als
\begin{minted}{haskell}
import Data.Time

data Datum = Datum
             { tag  :: Day
             , zeit :: TimeOfDay
             } deriving (Show, Eq)

> Datum (fromGregorian 2014 1 2) (TimeOfDay 13 37 0)
Datum {tag = 2014-01-02, zeit = 13:37:00}
\end{minted}

\end{frame}

\begin{frame}[fragile]
Eine IP (0.0.0.0 - 255.255.255.255) ist darstellbar als
\begin{minted}{haskell}
import Data.Word

data IP = IP Word8 Word8 Word8 Word8 deriving (Show,Eq)

> IP 13 37 13 37
IP 13 37 13 37
\end{minted}
mit Word8 als unsigned 8-Bit-Integer.\\
\pause
Wenn falsche Werte nicht darstellbar sind, dann haben wir sie auch nicht in unserem Programm als Problem.
\end{frame}

\begin{frame}[fragile]
Ein Gerät (String als Identifier) können wir nur als Solches definieren:
\begin{minted}{haskell}
data Geraet = Mouse 
            | Keyboard 
            | Monitor 
            | Speakers 
            deriving (Show,Eq)
\end{minted}
\pause
Hier können wir nachher bei Bedarf auch schnell welche hinzufügen und der Compiler meckert dann an allen stellen herum, wo wir nicht mehr alle Fälle abfangen.
\end{frame}

\subsection{Parsing in die Datenstrukturen}
\begin{frame}[fragile]
\begin{minted}{haskell}
> Datum (fromGregorian 2014 1 2) (TimeOfDay 13 37 0)
Datum {tag = 2014-01-02, zeit = 13:37:00}
\end{minted}
\pause
Mit attoparsec liest sich der Code fast von allein:
\begin{overprint}
\onslide<2>
\begin{minted}{haskell}
{-# LANGUAGE OverloadedStrings #-}
import Data.Time
import Data.Attoparsec.Char8

zeitParser :: Parser Datum
zeitParser = do
             undefined
\end{minted}
\onslide<3>
\begin{minted}{haskell}
{-# LANGUAGE OverloadedStrings #-}
import Data.Time
import Data.Attoparsec.Char8

zeitParser :: Parser Datum
zeitParser = do
  y  <- count 4 digit
  undefined
\end{minted}
\onslide<4>
\begin{minted}{haskell}
{-# LANGUAGE OverloadedStrings #-}
import Data.Time
import Data.Attoparsec.Char8

zeitParser :: Parser Datum
zeitParser = do
  y  <- count 4 digit
  char '-'
  undefined
\end{minted}
\onslide<5>
\begin{minted}{haskell}
{-# LANGUAGE OverloadedStrings #-}
import Data.Time
import Data.Attoparsec.Char8

zeitParser :: Parser Datum
zeitParser = do
  y  <- count 4 digit
  char '-'
  mm <- count 2 digit
  char '-'
  undefined
\end{minted}
\onslide<6>
\begin{minted}{haskell}
{-# LANGUAGE OverloadedStrings #-}
import Data.Time
import Data.Attoparsec.Char8

zeitParser :: Parser Datum
zeitParser = do
  y  <- count 4 digit; char '-'
  mm <- count 2 digit; char '-'
  undefined
\end{minted}
\onslide<7>
\begin{minted}{haskell}
{-# LANGUAGE OverloadedStrings #-}
import Data.Time
import Data.Attoparsec.Char8

zeitParser :: Parser Datum
zeitParser = do
  y  <- count 4 digit; char '-'
  mm <- count 2 digit; char '-'
  d  <- count 2 digit; char ' '
  h  <- count 2 digit; char ':'
  m  <- count 2 digit; char ':'
  s  <- count 2 digit
  undefined
\end{minted}
\onslide<8>
\begin{minted}{haskell}
{-# LANGUAGE OverloadedStrings #-}
import Data.Time
import Data.Attoparsec.Char8

zeitParser :: Parser Datum
zeitParser = do
  y  <- count 4 digit; char '-'
  mm <- count 2 digit; char '-'
  d  <- count 2 digit; char ' '
  h  <- count 2 digit; char ':'
  m  <- count 2 digit; char ':'
  s  <- count 2 digit
  return $
    Datum { tag  = fromGregorian (read y) (read mm) (read d)
          , zeit = TimeOfDay (read h) (read m) (read s)
          }
\end{minted}
\end{overprint}

\end{frame}

\begin{frame}[fragile]
Für die IP sieht der Code ähnlich aus:
\small
\begin{minted}{haskell}
{-# LANGUAGE OverloadedStrings #-}
import Data.Attoparsec.Char8
import Data.Word

parseIP :: Parser IP
parseIP = do
  d1 <- decimal
  char '.'
  d2 <- decimal
  char '.'
  d3 <- decimal
  char '.'
  d4 <- decimal
  return $ IP d1 d2 d3 d4
\end{minted}
\normalsize
\end{frame}

\begin{frame}[fragile]
Für das Gerät brauchen wir die Mächtigkeit von \texttt{Alternative}:
\small
\begin{minted}{haskell}
{-# LANGUAGE OverloadedStrings #-}
import Data.Attoparsec.Char8
import Control.Applicative

geraetParser :: Parser Geraet
geraetParser =
     (string "mouse"    >> return Mouse)
 <|> (string "keyboard" >> return Keyboard)
 <|> (string "monitor"  >> return Monitor)
 <|> (string "speakers" >> return Speakers)
\end{minted}
\normalsize
\pause
\texttt{<|>} ist der alternativ-Operator, der das Linke ausfürht und bei einem Fehler das Rechte zurückgibt.\\
\pause
Wir matchen hier jeweils auf den String, schmeissen ihn weg und liefern unseren Datentyp zurück.
\end{frame}

\begin{frame}[fragile]
Für die gesamte Zeile packen wir einfach unsere Parser zusammen:
\small
\begin{minted}{haskell}
zeilenParser :: Parser LogZeile
zeilenParser = do
     datum <- zeitParser
     char ' '
     ip <- parseIP
     char ' '
     geraet <- geraetParser
     return $ LogZeile datum ip geraet
\end{minted}
\normalsize
\end{frame}

\begin{frame}[fragile]
Für das Log nehmen wir die many-Funktion aus \glqq Alternative\grqq :
\begin{minted}{haskell}
logParser :: Parser Log
logParser = many $ zeilenParser <* endOfLine
\end{minted}
\pause
\texttt{<*} (aus Applicative) fungiert hier als ein Parser-Kombinator, der erst links matched, dann rechts matched und dann das Ergebnis des Linken zurückliefert.\\
\pause
Hier verwenden wir diesen um das Zeilenende hinter jeder Zeile loszuwerden.
\end{frame}

\begin{frame}[fragile]
Nun schreiben wir ein kleines Testprogramm:
\begin{minted}{haskell}
main :: IO ()
main = do
     log <- B.readFile "log.txt"
     print $ parseOnly logParser log
\end{minted}
und führen es aus:
\tiny
\begin{verbatim}
Right [LogZeile (Datum {tag = 2013-06-29, zeit = 11:16:23}) (IP 124 67 34 60) Keyboard,
       LogZeile (Datum {tag = 2013-06-29, zeit = 11:32:12}) (IP 212 141 23 67) Mouse,
       LogZeile (Datum {tag = 2013-06-29, zeit = 11:33:08}) (IP 212 141 23 67) Monitor,
       LogZeile (Datum {tag = 2013-06-29, zeit = 12:12:34}) (IP 125 80 32 31) Speakers,
       LogZeile (Datum {tag = 2013-06-29, zeit = 12:51:50}) (IP 101 40 50 62) Keyboard,
       LogZeile (Datum {tag = 2013-06-29, zeit = 13:10:45}) (IP 103 29 60 13) Mouse]
\end{verbatim}
\normalsize

\end{frame}

\section{Parser-Funktionsweise}
\begin{frame}[fragile]
Wie funktioniert so ein Parser nun intern genau?\\
\bigskip
\pause
Wir werden im folgenden \texttt{attoparsec} nicht im Detail erklären, sondern eher die Idee, die dahinter steht.
\end{frame}

\subsection{Poor-mans parser}
\begin{frame}[fragile]
\begin{overprint}
\onslide<1-2>
Eigentlich hätten wir gerne eine Funktion
\onslide<3-6>
Also brauchen wir
\end{overprint}
\begin{overprint}
\onslide<1-2>
\begin{minted}{haskell}
parse :: String -> a
\end{minted}
\onslide<3-4>
\begin{minted}{haskell}
parse :: String -> (a, String)
\end{minted}
\onslide<5>
\begin{minted}{haskell}
parse :: String -> (Maybe a, String)
\end{minted}
\onslide<6->
\begin{minted}{haskell}
parse :: String -> (Either String a, String)
\end{minted}
\end{overprint}
\begin{overprint}
\onslide<2>
aber was machen wir, wenn etwas von dem String überbleibt, damit wir später weitere dinge parsen können?
\onslide<4>
Aber das Parsen kann auch Fehlschlagen.
\onslide<5>
Und Fehlermeldungen wären auch schön.
\end{overprint}
\pause
\pause
\pause
\pause
\pause
\pause
Da wir auch so elegant die \texttt{do}-Notation nutzen möchten, brauchen wir eine Monade. Dazu brauchen wir einen einzelnen Typen:\\
\pause
\begin{minted}{haskell}
data Parser a = 
     Parser {
            runParser :: String -> (Either String a, String)
     }
\end{minted}
\pause
Dies gibt uns (wie schon beim State) 2 Funktionen:
\begin{minted}{haskell}
Parser    :: (String -> (Either String a, String)) -> Parser a
runParser :: Parser a -> String -> (Either String a, String)
\end{minted}
\end{frame}

\subsection{F/A/M}
\begin{frame}[fragile]
\begin{minted}{haskell}
Parser    :: (String -> (Either String a, String)) -> Parser a
runParser :: Parser a -> String -> (Either String a, String)
\end{minted}
\begin{overprint}
\onslide<1>
\begin{minted}{haskell}
instance Functor Parser where
fmap f p = Parser $ \input -> undefined
\end{minted}
\onslide<2>
\begin{minted}{haskell}
instance Functor Parser where
fmap f p = Parser $ \input ->
             let
               (res, rem) = runParser p input
             in
               undefined
\end{minted}
\onslide<3>
\begin{minted}{haskell}
instance Functor Parser where
fmap f p = Parser $ \input ->
             let
               (res, rem) = runParser p input
             in
               (fmap f res, rem)
\end{minted}
\end{overprint}
\end{frame}

\begin{frame}[fragile]
\begin{minted}{haskell}
Parser    :: (String -> (Either String a, String)) -> Parser a
runParser :: Parser a -> String -> (Either String a, String)
\end{minted}
\begin{overprint}
\onslide<1>
\begin{minted}{haskell}
instance Applicative Parser where
  pure a    = Parser $ \input -> undefined
  pf <*> pa = undefined
\end{minted}
\onslide<2>
\begin{minted}{haskell}
instance Applicative Parser where
  pure a    = Parser $ \input -> (Right a,input)
  pf <*> pa = undefined
\end{minted}
\onslide<3>
\begin{minted}{haskell}
instance Applicative Parser where
  pure a    = Parser $ \input -> (Right a,input)
  pf <*> pa = Parser $ \input ->
                undefined
\end{minted}
\onslide<4>
\begin{minted}{haskell}
instance Applicative Parser where
  pure a    = Parser $ \input -> (Right a,input)
  pf <*> pa = Parser $ \input ->
                let
                  (rf, rem1) = runParser pf input
                  (ra, rem2) = runParser pa rem1
                in
                  undefined
\end{minted}
\onslide<5>
\begin{minted}{haskell}
instance Applicative Parser where
  pure a    = Parser $ \input -> (Right a,input)
  pf <*> pa = Parser $ \input ->
                let
                  (rf, rem1) = runParser pf input
                  (ra, rem2) = runParser pa rem1
                in
                  (rf <*> ra, rem2)
\end{minted}
\end{overprint}
\end{frame}

\begin{frame}[fragile]
\begin{minted}{haskell}
Parser    :: (String -> (Either String a, String)) -> Parser a
runParser :: Parser a -> String -> (Either String a, String)
\end{minted}
\begin{overprint}
\onslide<1>
\begin{minted}{haskell}
instance Monad Parser where
  return a  = pure a
  pa >>= f  = Parser $ \input ->
                undefined
\end{minted}
\onslide<2>
\begin{minted}{haskell}
instance Monad Parser where
  return a  = pure a
  pa >>= f  = Parser $ \input ->
                let
                  (ra, rem1) = runParser pa input
                in
                  undefined
\end{minted}
\onslide<3>
\begin{minted}{haskell}
instance Monad Parser where
  return a  = pure a
  pa >>= f  = Parser $ \input ->
                let
                  (ra, rem1) = runParser pa input
                in
                  case ra of
                    Left err -> undefined
                    Right a  -> undefined
\end{minted}
\onslide<4>
\begin{minted}{haskell}
instance Monad Parser where
  return a  = pure a
  pa >>= f  = Parser $ \input ->
                let
                  (ra, rem1) = runParser pa input
                in
                  case ra of
                    Left err -> (Left err, rem1)
                    Right a  -> undefined
\end{minted}
\onslide<5>
\begin{minted}{haskell}
instance Monad Parser where
  return a  = pure a
  pa >>= f  = Parser $ \input ->
                let
                  (ra, rem1) = runParser pa input
                in
                  case ra of
                    Left err -> (Left err, rem1)
                    Right a  -> runParser (f a) rem1
\end{minted}
\end{overprint}
\end{frame}

\begin{frame}[fragile]
Was fehlt uns noch?\\
\pause
\begin{itemize}
 \item Langweilige Definition simpler Parser (digit, lit, space, ...)
 \pause
 \item Erweiterung mit sog. Continuations (Text \glqq nachschieben\grqq)
 \pause
 \item Implementation einer Standardfehlerbehandlung
\end{itemize}
\pause
Wir kümmern uns zunächst um das letzgenannte, indem wir \texttt{Alternative} implementieren.
\end{frame}

\subsection{Alternative}
\begin{frame}[fragile]
\begin{minted}{haskell}
Parser    :: (String -> (Either String a, String)) -> Parser a
runParser :: Parser a -> String -> (Either String a, String)
\end{minted}
\begin{overprint}
\onslide<1>
\begin{minted}{haskell}
instance Alternative Parser where
  empty     = Parser $ \input -> undefined
  pa <|> pb = undefined
\end{minted}
\onslide<2>
\begin{minted}{haskell}
instance Alternative Parser where
  empty     = Parser $ \inp -> (Left "",inp)
  pa <|> pb = undefined
\end{minted}
\onslide<3>
\begin{minted}{haskell}
instance Alternative Parser where
  empty     = Parser $ \inp -> (Left "",inp)
  pa <|> pb = Parser $ \input -> 
                let
                  (ra, rem1) = runParser pa input
                in
                  undefined
\end{minted}
\onslide<4>
\begin{minted}{haskell}
instance Alternative Parser where
  empty     = Parser $ \inp -> (Left "",inp)
  pa <|> pb = Parser $ \input -> 
                let
                  (ra, rem1) = runParser pa input
                in
                  case ra of
                    Right _ -> (ra, rem1)
                    _       -> runParser pb input
\end{minted}
Schlägt der Parser hier fehl, dann wird der bereits benutzte Input wiederhergestellt und ein weiterer Parser probiert.
\end{overprint}
\end{frame}

\subsection{Verwendung}

\begin{frame}[fragile]
Nun können wir ein paar Simple Dinge parsen:
\begin{minted}{haskell}
parse1 :: Parser Int
parse1 = Parser $ \inp -> case inp of
                    ('1':xs) -> (Right 1,xs)
                    x        -> (Left "no 1", x)

parse0 :: Parser Int
parse0 = Parser $ \inp -> case inp of
                    ('0':xs) -> (Right 0, xs)
                    x        -> (Left "no 0", x)
\end{minted}
\pause
Dies sind dann intrinsics oder interne Funktionen, die ohne einen Zugriff auf \glqq Parser\grqq \ nicht möglich sind.\\
\pause
Man versteckt die Funktion \texttt{Parser}, damit niemand anderes solche \glqq intrinsics\grqq \ definieren kann. Damit wird eine manipulation des Inputs bzw. ein fehlerhaftes Anwenden unmöglich.
\end{frame}

\begin{frame}[fragile]
Binärzahlen können wir dann wie folgt parsen:
\begin{overprint}
\onslide<1>
\begin{minted}{haskell}
parseBin :: Parser Int
parseBin = parse0 <|> parse1
\end{minted}
\onslide<2>
\begin{minted}{haskell}
parseBin :: Parser Int
parseBin = parse0 <|> parse1

testParser :: Parser [Int]
testParser = many parseBin
\end{minted}
\onslide<3>
\begin{minted}{haskell}
parseBin :: Parser Int
parseBin = parse0 <|> parse1

testParser :: Parser [Int]
testParser = many parseBin

main :: IO ()
main = print $ runParser testParser "101001"
-- (Right [1,0,1,0,0,1],"")
\end{minted}
\end{overprint}
\end{frame}

\begin{frame}[fragile]
Alle gängigen Parsing-Bibliotheken benutzt man ähnlich. Die Bibliotheken stellen minimalste Intrinsics zur verfügung um z.B.
\pause
\begin{itemize}
 \item Zeichen zu parsen
 \pause
 \item solange Zeichen zu parsen, bis ein Stoppzeichen kommt
 \pause
 \item Ziffern zu parsen
 \pause
 \item Ziffern zu parsen, bis non-Ziffern kommen und die Zahl zurückgeben
 \pause
 \item etc.
\end{itemize}
\pause
Aus diesen Kleinstbausteinen konstruiert man dann Parser für größere Datenstrukturen.\\\par
\pause
So haben wir in unserem Log-Beispiel zunächst nur Teil-Parser definiert (Datum, IP, Gerät) und diese dann einfach kombiniert.
\end{frame}

\end{document}