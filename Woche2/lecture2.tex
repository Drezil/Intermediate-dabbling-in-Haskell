\documentclass{beamer}

\usepackage[utf8]{inputenc}
\usepackage[T1]{fontenc}
\usepackage[ngerman]{babel}
\usepackage{graphicx} % Bilder
\usepackage{wrapfig} % Umflussbilder
\usepackage{multicol} % Multiple columns
\usepackage{minted} % Haskell source code
\usepackage{framed} % Frames around source code
\usepackage[framemethod=tikz]{mdframed} % Frames
\usepackage{verbatim} % \begin{comment}...\end{comment}
\usepackage{etoolbox} % manipulate minted
\AtBeginEnvironment{minted}{\fontsize{10}{10}\selectfont}
\AfterEndEnvironment{minted}{}

\mdfdefinestyle{fancy}{
  roundcorner=5pt,
  linewidth=4pt,
  linecolor=red!80,
  backgroundcolor=red!20
}
\newmdenv[style=fancy]{important}

% redifine \em for \emph to use bold instead of italics
\makeatletter
\DeclareRobustCommand{\em}{%
  \@nomath\em \if b\expandafter\@car\f@series\@nil
  \normalfont \else \bfseries \fi}
\makeatother

% Stuff for Beamer
\beamertemplatenavigationsymbolsempty
\usetheme{Warsaw}

\title{Fortgeschrittene funktionale Programmierung in Haskell}

\begin{document}

%  \usebackgroundtemplate{\includegraphics[width=\paperwidth,height=\paperheight]{1.jpg}} 
  
%----------------------------------------------------------------------------------------  

  \begin{frame}
  \begin{center}
    \huge\textbf{Fortgeschrittene funktionale Programmierung in Haskell}\\ \bigskip
    \LARGE Universität Bielefeld, Sommersemester 2015\\ \bigskip
    \large Jonas Betzendahl \& Stefan Dresselhaus
    \end{center}
  \end{frame}

%----------------------------------------------------------------------------------------  
\begin{frame}[allowframebreaks]{Outline}
\frametitle{Übersicht}
\tableofcontents[hideallsubsections]
\end{frame}

\section{Typen}
\subsection{Beispiele}
\begin{frame}
 Primitive Datentypen sind eine Annotation, wie die Bits im Speicher interpretiert werden sollen.\\
 \pause
 Einige primitive Datentypen sollten euch aus anderen Programmiersprachen schon bekannt sein:
 \begin{itemize}
  \item Zahlen (z.B. Int, Integer, Float, Double, ...)
  \item Zeichenketten (z.B. String, UTF-8-Strings, ...)
  \item Bool
 \end{itemize}
\end{frame}

\begin{frame}
 Es gibt auch Datentypen höherer Ordnung. Diese zeichnen sich dadurch aus, dass sie alleine nicht vollständig sind.\\
 \pause
 Auch hier sollten schon einige bekannt sein:\\
 (a,k,v steht hier jeweis für einen (primitiven) Datentypen)
 \begin{itemize}
  \item Liste von a
  \item Hashmap von k und v
  \item Vektor von a
  \item Tree von a
  \item Zusammengesetzte Typen (z.B. Structs in C/C++)
 \end{itemize}
 \pause
 Im folgenden gehen wir auf 2 wesentliche zusammengesetzte Typen in Haskell ein: Maybe und Either.
\end{frame}


\subsection{Maybe}
\begin{frame}[fragile]
 Einen neuen Datentypen definieren wir in Haskell mit dem Keyword \texttt{data}:
 \begin{minted}{haskell}
 data Maybe a = Nothing
              | Just a
 \end{minted}
 \pause
 Was hat das für einen Sinn?\\
 \pause
 Maybe gibt das Ergebnis einer Berechnung an, die fehlschlagen kann.\\
 In klassischen Sprachen wird hier meist ein \glqq abgesprochener\grqq \ Fehlerzustand zurückgegeben (0, -1, null, ...). In Haskell wird dies über den Rückgabetyp deutlich gemacht.
\end{frame}

\begin{frame}[fragile]
 Nachteile
 \begin{itemize}
  \item Ein neuer Datentyp, den man kennen muss
 \end{itemize}
 \pause
 Vorteile
 \begin{itemize}
  \item keine Absprachen, die man vergessen kann
  \item einheitliche Behandlung aller Fälle
  \item mehrere möglicherweise fehlschlagende Operationen gruppieren und nur solange evaluieren, bis die erste fehlschlägt oder alle erfolgreich sind
 \end{itemize}
\end{frame}

\begin{frame}[fragile]
 Beispiel: Finden eines Elementes in einer Liste
 \pause
 \begin{minted}{haskell}
 ghci > import Data.List
 ghci > findIndex (== 5) [1..10]
 Just 4               -- [1..10] !! 4 => 5
 
 ghci > findIndex (== 1000) [1..10]
 Nothing
 \end{minted}
 \pause
 Da wir \texttt{1000} in der Liste der Zahlen von 1-10 nicht finden können, haben wir keinen gültigen Index, daher bekommen wir ein Nothing.
\end{frame}

\subsection{Either}

\begin{frame}[fragile]
 \begin{minted}{haskell}
 data Either a b = Left a
                 | Right b
 \end{minted}
 \pause
 Was hat das für einen Sinn?\\
 \pause
 Either benutzt man, wenn man ein erwartetes Ergebnis \texttt{Right b} vom Typen \texttt{b} hat \textbf{oder} einen Fehlerzustand \texttt{Left a} vom Typen \texttt{a}.\\
 Meistens ist das erste Argument \texttt{String} um eine lesbare Fehlermeldung zu bekommen.\\
 \pause
 \bigskip
 Einfach zu merken: \glqq Right\grqq \ ist der \glqq richtige\grqq \ Fall.
\end{frame}

\begin{frame}[fragile]
 Beispiele für eine Benutzung von Either:
 \begin{minted}{haskell}
 parse5 :: String -> Either String Int
 parse5 "5" = Right 5
 parse5 _   = Left "Could not parse 5"
 
 parse5 "5"   -- Right 5
 parse5 "abc" -- Left "Could not parse 5"
 \end{minted}
\end{frame}

\subsection{Produkte, Summen und Exponenten}
\begin{frame}[fragile]
Man kann Typen generell in 4 Kategorien einteilen:
\pause
\begin{itemize}
 \item Summentypen: Maybe, Either, Bool, Int, ...
 \item Produkttypen: Vektoren, Tupel, ...
 \item rekursive Typen: Liste, Stream, ...
 \item Exponentialtypen: Funktionen
\end{itemize}
\end{frame}

\begin{frame}[fragile]
Wie viele mögliche Werte kann
\begin{minted}{haskell}
data Sum = S1 Bool | S2 (Maybe Bool) deriving Show
\end{minted}
annehmen?
\pause
\begin{minted}{haskell}
S1 True | S1 False |
S2 (Just True) | S2 (Just False) | S2 (Nothing)
\end{minted}
Insgesamt 5. Daher auch die Bezeichnung Summentyp, da $2+(2+1) = 5$ (2 Belegungen für Bool, 2+1 für Maybe Bool).
\end{frame}

\begin{frame}[fragile]
Wie viele mögliche Werte kann
\begin{minted}{haskell}
data Prod = P Bool Sum
\end{minted}
annehmen?
\pause
\begin{minted}{haskell}
P True (S1 True) | P False (S1 True) | P True (S1 False) | ...
... | P1 True (S2 Nothing) | P1 False (S2 Nothing)
\end{minted}
Insgesamt 10. Daher auch die Bezeichnung Produkttyp, da $2*5 = 10$ (2 Belegungen für Bool, 5 für Sum).
\end{frame}

\begin{frame}[fragile]
Wie viele mögliche Funktionen der Form
\begin{minted}{haskell}
f :: Maybe Bool -> Bool
\end{minted}
gibt es?
\pause
Insgesamt 8, weil $8 = 2^3$:
\begin{verbatim}
Just True | Just False | Nothing
    False        False     False
    False        False      True
    False         True     False
    False         True      True
     True        False     False
     True        False      True
     True         True     False
     True         True      True
\end{verbatim}
\end{frame}

\begin{frame}[fragile]
Wozu interessiert uns das?
\pause
\begin{itemize}
 \item Wir können Abschätzen, wie viele \textbf{mögliche} Implementationen es gibt
 \item Es sollte uns lehren immer die kleinstmöglichen Typen zu nehmen
 \item Es bietet viele theoretische Spielereien (Beweise, Kategorientheorie, ...)
 \pause
 \item Wir können euch jetzt mit \glqq Summentyp\grqq \ und \glqq Produkttyp\grqq \ nerven und können hierauf verweisen :)
\end{itemize}

\end{frame}

% ------------------------------------------------------------------------------------------------------------

\section{Typklassen}
\begin{frame}
Viele Typen haben ähnliche oder gleiche Eigenschaften. Diese Eigenschaften fasst man zu Typklassen zusammen.\\
\pause
\begin{itemize}
 \item Zahlen kann man alle verrechnen, auch wenn z.B. Int und Double verschiedene Typen haben
 \item Listen, Vektoren, Arrays haben alle Elemente, über die man z.B. iterieren kann
 \item Maybe, Either, Listen, etc. haben (vielleicht) Elemente, die man verändern kann
\end{itemize}
\pause
\begin{important}
\textbf{Warnung:} Typklassen haben nichts mit den Klassen der Objektorientierung zu tun, sondern eher mit Templates und abstrakten Klassen
\end{important}
\end{frame}

\subsection{Beispiele}
\begin{frame}[fragile]
\begin{minted}{haskell}
class  Eq a  where
     (==) :: a -> a -> Bool
--or (/=) :: a -> a -> Bool

class Eq a => Ord a  where
     (<=) :: a -> a -> Bool
-- definiert automatisch: compare, >=, <, >, max, min
\end{minted}
\pause
Im folgenden stellen wir 3 Zentrale Typklassen vor: Functor, Applicative, Monad
\end{frame}

% ------------------------------------------------------------------------------------------------------------

\subsection{Functor}
\begin{frame}[fragile]
 Ein Funktor F lässt sich auf jedem Datentypen definieren, der sowas wie einen \glqq Inhalt\grqq \ hat.\\
 \pause
 Genauer: Er wird definiert über die Funktion \texttt{fmap}, die es erlaubt eine Funktion auf den Inhalt anzuwenden.
 \begin{minted}{haskell}
class Functor f where
    fmap :: (a -> b) -> f a -> f b
 \end{minted}
 \texttt{f} heisst hier \textbf{der Kontext}, in dem \texttt{a} existiert.\\
\pause
\bigskip
Spoiler: Liste ist ein Funktor. Maybe auch. Wieso?
\end{frame}

\begin{frame}[fragile]
Hilfreiche Analogie:\\
Für den Anfang kann man sich den \textbf{Kontext} als eine Kiste vorstellen, in der etwas liegt.\\
\bigskip
\pause
\texttt{fmap} wertet also nur Funktionen, die auf dem \textbf{Inhalt} der Kiste funktionieren würden, zu Funktionen auf, die auf \textbf{Kisten mit Dingen} funktionieren.\\\bigskip
\pause
Man kann \texttt{fmap} daher auch etwas anders Klammern:
\begin{minted}{haskell}
    fmap :: (a -> b) -> (f a -> f b)
\end{minted}
\texttt{fmap} nimmt somit eine Funktion und gibt eine neue Funktion zurück, die auf dem Kontext \texttt{f} funktioniert.
\end{frame}

\begin{frame}[fragile]
Functor-Instanz von Maybe:
\begin{minted}{haskell}
instance Functor Maybe where
    fmap f (Just a) = Just (f a)
    fmap _ Nothing  = Nothing
\end{minted}
\begin{comment}
%ÜBUNG
\pause
Functor-Instanz von Either:
\begin{minted}{haskell}
instance Functor Either where
    fmap f (Right a) = Right (f a)
    fmap _ (Left b)  = Left b
\end{minted}
\end{comment}
\pause
Functor-Instanz von Listen:
\begin{minted}{haskell}
instance Functor [] where
    fmap f (x:xs) = f x : (fmap f xs)
    fmap _ []     = []
\end{minted}
\texttt{fmap} auf Listen ist einfach das bekannte \texttt{map}.\\
\pause
\texttt{fmap} hat auch eine infix-Schreibweise: \texttt{<\$>}.
\end{frame}


\begin{frame}[fragile]
Beispiel:
\mint{haskell}|ghci > fmap (+1) (Just 3)|
\pause
\mint{haskell}|       Just 4|
\mint{haskell}|ghci > fmap (+1) Nothing|
\pause
\mint{haskell}|       Nothing|
\mint{haskell}|ghci > fmap (+1) [1..10]|
\pause
\mint{haskell}|       [2,3,4,5,6,7,8,9,10,11]|
\mint{haskell}|ghci > fmap (+1) (Right 2)|
\pause
\mint{haskell}|       Right 3|
\mint{haskell}|ghci > fmap (+1) (Left 2)|
\pause
\mint{haskell}|       Left 2|
\end{frame}

\begin{frame}[fragile]
Funktoren sind mathematische Objekte mit Eigenschaften. Damit der Compiler auch die richtigen Optimierungen machen kann, muss jeder Funktor folgende Regeln erfüllen:\\
\pause
\begin{minted}{haskell}
-- Strukturerhaltung
fmap id = id
\end{minted}
Die Datenstruktur darf sich nicht ändern.\\
\pause
\bigskip
\begin{minted}{haskell}
-- Composability
fmap (f . g) = fmap f . fmap g
\end{minted}
Mehrere \texttt{fmap}s hintereinander dürfen zusammengefasst werden, ohne, dass sich das Ergebnis ändert.
\end{frame}

\begin{frame}[fragile]
\begin{important}\begin{center}\textbf{Negativbeispiel}\end{center}\end{important}
\begin{minted}{haskell}
fmap' f []     = []
fmap' f (a:as) = (f a):a:(fmap' f as)
\end{minted}
\pause
\begin{minted}{haskell}
fmap' id [1,2,3] = [1,1,2,2,3,3] /= [1,2,3]
\end{minted}
\end{frame}

\begin{frame}[fragile]
\begin{important}\begin{center}\textbf{Negativbeispiel}\end{center}\end{important}
\begin{minted}{haskell}
fmap' f Nothing  = Nothing
fmap' f (Just a) = Just (f (f a))
\end{minted}
\pause
\begin{minted}{haskell}
(fmap' (+1) . fmap' (*2)) (Just 1)
= fmap' (+1) (Just ((*2) ((*2) 1))
= fmap' (+1) (Just 4)
= Just 6

(fmap' ((+1).(*2)) (Just 1) 
= Just (((+1).(*2).(+1).(*2)) 1)
= Just 7
\end{minted}

\end{frame}

% ------------------------------------------------------------------------------------------------------------

\subsection{Applicative}
\begin{frame}[fragile]
Applicative funktioniert ähnlich zu Funktor. Hierbei kann man auch mit Funktionen in einem Kontext arbeiten.\\
\pause
\begin{minted}{haskell}
class Functor f => Applicative f where
    pure  :: a -> f a
    (<*>) :: f (a -> b) -> f a -> f b
\end{minted}
\pause
\texttt{pure} bringt etwas in den Standardkontext (z.B. Liste mit 1 Element, Just x, Right x, ..).\\
\texttt{<*>} ist fast ein \texttt{fmap}, nur dass die Funktion auch in demselben Kontext liegt.
\end{frame}

\begin{frame}[fragile]
Applicative-Instanz von Maybe:
\begin{minted}{haskell}
instance Applicative Maybe where
    pure                   = Just
    Nothing  <*> _         = Nothing
    (Just f) <*> something = fmap f something
\end{minted}
\begin{comment}
% ÜBUNG
\pause
Applicative-Instanz von Either:
\begin{minted}{haskell}
instance Applicative Either where
    pure                    = Right
    (Right f) <*> something = fmap f something
    (Left e)  <*> _         = Left e
\end{minted}
\end{comment}
\pause
Applicative-Instanz von Listen:
\begin{minted}{haskell}
instance Applicative [] where
    pure a       = [a]
    (f:fs) <*> x = fmap f x ++ (fs <*> x)
    []     <*> _ = []
\end{minted}
\end{frame}

\begin{frame}[fragile]
Auch für Applicative gibt es Gesetze:
\begin{minted}{haskell}
pure f   <*> x      = fmap f x
pure id  <*> v      = v
pure f   <*> pure x = pure (f x)
     u   <*> pure y = pure ($ y) <*> u
pure (.) <*> u <*> v <*> w = u <*> (v <*> w)
\end{minted}
Diese Gesetze regeln (abgesehen von der Verkettung) nur, dass sich Applicative konsistent zu Functor verhält.
\end{frame}


\begin{frame}[fragile]
\mint{haskell}|ghci > import Control.Applicative|
\mint{haskell}|ghci > Just (+1) <*> Just 3|
\pause
\mint{haskell}|       Just 4|
\mint{haskell}|ghci > Nothing <*> Just 3|
\pause
\mint{haskell}|       Nothing|
\mint{haskell}|ghci > pure (+1) <*> Right 2|
\pause
\mint{haskell}|       Right 3|
\mint{haskell}|ghci > pure (+1) <*> Just 2|
\pause
\mint{haskell}|       Just 3|
\mint{haskell}|ghci > [(+1),(*2)] <*> [1..5]|
\pause
\mint{haskell}|       [2,3,4,5,6,2,4,6,8,10]|
\mint{haskell}|ghci > pure (*) <*> [1..3] <*> [1..3]|
\pause
\mint{haskell}|       [1,2,3,2,4,6,3,6,9]|
\end{frame}

% ------------------------------------------------------------------------------------------------------------

\subsection{Monoid}
\begin{frame}[fragile]
Ein Monoid ist ein Konzept, was häufiger mal auftaucht. Mathematisch gesehen ist ein Monoid eine Menge mit \textbf{assoziativer Operation} und \textbf{neutralem Element}.\\
Folglich lautet die Definition in Haskell:
\begin{minted}{haskell}
class Monoid a where
 mempty  :: a
 mappend :: a -> a -> a
 mconcat :: [a] -> a
 mconcat = foldr mappend mempty
\end{minted}
\pause
Monoide, die ihr schon kennt, sind z.B. Listen (mit \texttt{[]} und \texttt{++}) und Bäume.\\
\pause
Für \texttt{mappend} schreibt man auch \texttt{<>}.
\end{frame}

% ------------------------------------------------------------------------------------------------------------

\subsection{Monad}
\begin{frame}
Wozu Monaden?
\pause
\begin{itemize}
 \item Man kann auch ohne funktional programmieren
 \pause
 \item Monaden verhalten sich wie ein Semikolon in anderen Sprachen
 \pause
 \item Monaden \glqq arbeiten\grqq \ im Hintergrund
\end{itemize}

\end{frame}

\begin{frame}[fragile]
Beispiel
\begin{minted}{haskell}
f            :: Maybe Header
getInbox     :: Maybe Folder
getFirstMail :: Folder -> Maybe Mail
getHeader    :: Mail -> Maybe Header
\end{minted}
\pause
Ohne Monaden:
\begin{minted}{haskell}
f = case getInbox of
      (Just folder) -> 
          case getFirstMail folder of
            (Just mail) -> 
                case getHeader mail of
                  (Just head) -> return head
                  Nothing     -> Nothing
            Nothing     -> Nothing
      Nothing       -> Nothing
\end{minted}
\end{frame}
\begin{frame}[fragile]
Beispiel
\begin{minted}{haskell}
f            :: Maybe Header
getInbox     :: Maybe Folder
getFirstMail :: Folder -> Maybe Mail
getHeader    :: Mail -> Maybe Header
\end{minted}
Mit Monaden:
\begin{minted}{haskell}
f = do
      folder <- getInbox
      mail   <- getFirstMail folder
      header <- getHeader mail
      return header
\end{minted}

\end{frame}

\begin{frame}[fragile]
Wie funktioniert diese Magie?\\
\pause
Monaden benutzen die Funktion \glqq bind \grqq \ (\texttt{>>=}) um Berechnungen zu verketten und arbeit für uns im Hintergrund zu übernehmen.\\
\pause
\begin{minted}{haskell}
class Applicative m => Monad m where
  return :: a -> m a
  (>>=)  :: m a -> (a -> m b) -> m b
\end{minted}
\pause
\texttt{return} funktioniert analog zu \texttt{pure} und bringt ein Element in den Standardkontext.\\
\texttt{>>=} enthält die ganze Magie. Prinzipiell \glqq packt\grqq \ es ein \texttt{m a} aus und wendet die mitgegegbene Funktion an.
\end{frame}

\begin{frame}[fragile]
Monad-Instanz von Maybe:
\begin{minted}{haskell}
instance Monad Maybe where
    return                = pure   -- = Just
    (Just a) >>= f        = f a
    Nothing >>= _         = Nothing
\end{minted}
\begin{comment}
% ÜBUNG
\pause
Monad-Instanz von Either:
\begin{minted}{haskell}
instance Monad Either where
    return                = pure   -- = Right
    (Right e) >>= f       = f e
    (Left e)  >>= _       = Left e
\end{minted}
\end{comment}
\pause
Monad-Instanz von Listen:
\begin{minted}{haskell}
instance Monad [] where
    return       = pure
    (x:xs) >>= f = f x ++ (xs >>= f)
    []     >>= _ = []
\end{minted}
\end{frame}

\begin{frame}[fragile]
Zurück zu unserem Beispiel. Wie helfen nun Monaden?
\pause
\begin{minted}{haskell}
f = case getInbox of
      (Just folder) -> 
          case getFirstMail folder of
            (Just mail) -> 
                case getHeader mail of
                  (Just head) -> return head
                  Nothing     -> Nothing
            Nothing     -> Nothing
      Nothing       -> Nothing
\end{minted}
\end{frame}

\begin{frame}[fragile]
Schreiben wir mittels \texttt{>>=} um zu:
\pause
\begin{minted}{haskell}
f = getInbox >>= (\folder ->
        getFirstMail folder >>= (\mail ->
            getHeader mail >>= (\header ->
                return header
            )
        )
    )
\end{minted}
\pause
\texttt{>>=} fäng hier den Nothing-Fall ab und wir geben eine Funktion mit, die nur noch den Just-Fall behandeln muss.
\end{frame}

\subsection{do-notation}
\begin{frame}[fragile]
Da dieses ganze \texttt{x >>= (\textbackslash v -> ... >>= (\textbackslash w -> ...))} hässlich ist, gibt es die do-notation.\\
Wir können das Beispiel von oben also umschreiben als:
\pause
\begin{minted}{haskell}
f = do
      folder <- getInbox
      mail   <- getFirstMail folder
      header <- getHeader mail
      return header
\end{minted}
\pause
\texttt{<-} extrahiert hier den Wert \glqq aus der Monade\grqq :\\
\mint{haskell}|getInbox :: Maybe Folder|
\mint{haskell}|folder   :: Folder      |
\pause
\begin{important}Dieses automatische Zusammenfassen funktioniert nur, wenn \textbf{alle} Funktionen als letzten Wert etwas in derselben Monade zurückgeben.\end{important}
\end{frame}

\begin{frame}[fragile]
In manchen Fällen können wir sogar auf die do-notation verzichten. Wenn jeder zurückgegebene Wert direkt von der nächsten Funktion gebraucht wird, dann sparen wir uns die lambda-Ausdrücke und verketten direkt:
\pause
\begin{minted}{haskell}
f = getInbox >>= getFirstMail >>= getHeader
\end{minted}
\pause
Eine weitere Funktion, die einem in diesem Kontext begegnet ist \texttt{>>}, welche das Ergebnis verwirft und die nächste Funktion ohne Parameter aufruft:
\begin{minted}{haskell}
main = putStrLn "Geben sie etwas ein."
                    >>  getLine 
                    >>= putStrLn
\end{minted}
\pause
Dieses Programm gibt einen String aus (mit dem Ergebnis \texttt{IO ()}) und wir schmeissen das Ergebnis weg und rufen einfach die nächsten Funktionen auf.
\end{frame}

\subsection{Monad-Rules}
\begin{frame}[fragile]
Monaden haben ähnlich zu Functor und Applicative auch mathematische Regeln, die man erfüllen Muss.\\
\begin{itemize}
 \item Linksidentität\\
       \mint{haskell}|return x >>= f == f x|
 \item Rechtsidentität\\
       \mint{haskell}|m >>= return == m|
 \item Assoziativität\\
       \mint{haskell}|(m >>= f) >>= g == m >>= (\x -> f x >>= g)|
\end{itemize}
\pause
Die Assoziativität ist etwas schwer zu erkennen. Deutlicher wird es, wenn wir eine umgeformte Funktion definieren:
\begin{minted}{haskell}
(<=<) :: (Monad m) => (b -> m c) -> (a -> m b) -> (a -> m c)
f <=< g = (\x -> g x >>= f)
\end{minted}
\end{frame}

\begin{frame}[fragile]
Somit gibt sich für die Regeln:
\begin{minted}{haskell}
return <=< f    == f
f <=< return    == f
(f <=< g) <=< h == f <=< (g <=< h)
\end{minted}
\pause
Diese Regeln sind relativ \glqq natürlich\grqq , da sie im prinzip nur Funktionskomposition \texttt{(.)} auf Monaden wohldefinieren.
\end{frame}

\section{Praktische Arbeit}

\subsection{do-notation}
\begin{frame}[fragile]
Die do-notation hängt - wie wir eben gesehen haben - mit der Monade zusammen und ist über \texttt{>>=} definiert.\\
\pause
Damit verhält sich diese Notation \textbf{für jede Monade anders}.\\
\pause
Vorher hatten wir nur \texttt{do} im Kontext von \texttt{IO}. Daher gehen wir nun auf andere Monaden ein und wie hier die do-notation genutzt wird.
\end{frame}


\subsection{List-Comprehension}
\begin{frame}[fragile]
Die bereits bekannte List-Comprehension
\begin{minted}{haskell}
let l = [x*y | x <- [1..5], y <- [1..5], x + y == 5]
\end{minted}
ist nur syntaktischer Zucker für die monadische do-notation:
\begin{minted}{haskell}
let l = do
        x <- [1..5]
        y <- [1..5]
        guard (x + y == 5)
        return x*y
\end{minted}
mit
\begin{minted}{haskell}
guard :: (MonadPlus m) => Bool -> m ()
guard True = return ()
guard False = mzero
\end{minted}

\end{frame}

\subsection{State-Monade}
\begin{frame}[fragile]
Häufig hat man das Problem, dass man einen Zustand in einem Programm herumreichen möchte.\\
\pause
Hierzu gibt es 2 Möglichkeiten:
\begin{enumerate}
 \item Den Zustand immer explizit an die Funktion übergeben
 \item Den Zustand in einer Monade verstecken
\end{enumerate}
\pause
Letzteres hat den Vorteil, dass man auch Funktionen aufwerten kann, die den Zustand ignorieren.
\end{frame}

\begin{frame}[fragile]
Die State-Monade macht keine \glqq Magie\grqq \ im Hintergrund, sondern hält einfach nur einen Zustand fest und hat den Typen
\mint{haskell}|State s a|
wobei \texttt{s} der interne Zustand und \texttt{a} der Rückgabewert ist.\\
\pause
Um damit zu arbeiten gibt es u.A. folgende Hilfsfunktionen:
\begin{minted}{haskell}
get    :: State s s
put    :: s -> State s ()
modify :: (s -> s) -> State s ()
\end{minted}
jeweils um den internen Zustand zu holen, setzen oder zu modifizieren.
\end{frame}

\begin{frame}[fragile]
Beispiel:
\begin{minted}{haskell}
countme :: a -> State Int a
countme a = do
              modify (+1)
              return a

example :: State Int Int
example = do
              x <- countme (2+2)
              y <- return (x*x)
              z <- countme (y-2)
              return z

examplemain = runState example 0
-- -> (14,2), 14 = wert von z, 2 = interner counter
\end{minted}
\end{frame}

\begin{comment}
\begin{frame}[fragile]
Beispiel 2:
\begin{minted}{haskell}
data Spieler = X | O
data TicTacToeSpielfeld = Array (Int,Int) Spieler
data TicTacToe a = State TicTacToeSpielfeld a

macheSpielzug :: (Int, Int) -> Spieler -> TicTacToe ()
macheSpielzug pos s = modify ((flip (//)) [(pos,s)])

spielVorbei :: TicTacToe Bool
spielVorbei = do
                spielfeld <- get
                return (istVorbei spielfeld)
\end{minted}
\end{frame}
\end{comment}

\begin{frame}[fragile]
Beispiel 2:
\begin{minted}{haskell}
module Main where
import Control.Monad.State
type CountState = (Bool, Int)
 
startState :: CountState
startState = (False, 0)

play :: String -> State CountState Int
--play  ...

\end{minted}
\end{frame}

\begin{frame}[fragile]
\begin{minted}{haskell}
play []     = do
              (_, score) <- get
              return score
play (x:xs) = do
 (on, score) <- get
 case x of
   'C' -> if on then put (on, score + 1) else put (on, score)
   'A' -> if on then put (on, score - 1) else put (on, score)
   'T' -> put (False, score)
   'G' -> put (True, score)
   _   -> put (on, score)
 play xs

main = print $ runState (play "GACAACTCGAAT") startState
-- -> (-3,(False,-3))
\end{minted}
\end{frame}




\end{document}