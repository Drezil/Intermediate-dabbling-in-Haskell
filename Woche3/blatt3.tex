\documentclass[a4paper,10pt]{scrartcl}
\usepackage[utf8]{inputenc}
\usepackage[ngerman]{babel}
\usepackage{amsmath}
\usepackage{amsfonts}
\usepackage{amsthm}
\usepackage{geometry}
\usepackage{multicol}
\usepackage{minted}
\geometry{a4paper,left=30mm,right=30mm, top=3cm, bottom=2cm} 

\newcommand{\underfat}[1]{\underline{\textbf{#1}}}
\newcommand{\theuebungszettel}{5}

\parindent0pt

\begin{document}

\begin{center}
  \begin{huge}
    \underfat{Fortgeschrittene Funktionale}\\
    \underfat{Programmierung in Haskell}\\
  \end{huge}
\begin{LARGE}
\textbf{Übungszettel \theuebungszettel}
\end{LARGE}
\end{center}
\section*{Aufgabe \theuebungszettel.1:}
In der Vorlesung wurde \texttt{MaybeT} (basierend auf der \texttt{Maybe}-Monade) explizit vorgestellt. Erstellen Sie analog einen Monadentransformer \texttt{EitherT}, basierend auf der \texttt{Either}-Monade.\smallskip

Zur Erinnerung: \texttt{Either} ist definiert als:
\begin{minted}{haskell}
data Either a b = Left a  -- "Fehler"
                | Right b -- "Erfolg"
\end{minted}
Erstellen Sie hierzu folgende Instanzen:
\begin{itemize}
 \item \texttt{Functor}
       \begin{minted}{haskell}
  instance Functor f => Functor (EitherT l f) where
      fmap :: (a -> b) -> (EitherT l f) a -> (EitherT l f) b
       \end{minted}
 \item \texttt{Applicative}
       \begin{minted}{haskell}
  instance Applicative f => Applicative (EitherT l f) where
      pure  :: a -> f a
      (<*>) :: Either l f (a -> b) -> Either l f a -> Either l f b
       \end{minted}
 \item \texttt{Monad}
       \begin{minted}{haskell}
  instance Monad m => Monad (EitherT l m) where
      return :: a -> (EitherT l m) a
      (>>=)  :: (EitherT l m) a -> (a -> (EitherT l m) b) -> (EitherT l m) b
       \end{minted}
\end{itemize}
Beispielcode mit Definitionen und Testfällen finden Sie in der Datei \texttt{eitherT.hs}. Nutzen Sie diese als Ausgangsbasis.

\section*{Aufgabe \theuebungszettel.2:}

Für die folgenden Aufgaben benötigen Sie die externe Bibliothek \texttt{mtl}\footnote{https://hackage.haskell.org/package/mtl}, in der der \texttt{RWST}-Stack (Read-Write-State-Transformer, siehe Vorlesung) bereits implementiert ist.\bigskip

Richten Sie sich hierzu mit dem Programm \texttt{cabal} in einem Verzeichnis eine lokale Arbeitsumgebung (genannt \emph{sandbox}) ein, indem Sie folgende Befehle verstehen und anschließend ausführen:
\begin{minted}{bash}
$ git init                            # git initialisieren - falls gewuenscht.
                                      # alternativ: mit "git clone" ein bestehendes
                                      # Repository klonen. Wir helfen gerne dabei.
$ cabal update                        # Paketliste aktualisieren
$ cabal init                          # Erstellen eines Paketes
$ cabal sandbox init                  # Initialisieren der Sandbox
$ nano <projektname>.cabal            # Hinzufuegen von mtl > 2.2.0 && < 2.3
                                      # und transformers >= 0.4.3 && < 0.4.4
                                      # als Dependency
                                      # Einstellen der Main durch aendern von
                                      #      main-is: game.hs
$ cabal install --only-dependencies   # Installieren aller Dependencies
$ cabal build                         # Projekt bauen
$ cabal run                           # Projekt ausfuehren
$ cabal repl                          # Einen ghci laden, in dem alle Dependencies
                                      # bereits geladen wurden
\end{minted}
%$

\section*{Aufgabe \theuebungszettel.3:}
In dieser Aufgabe geht es um die Verwendung eines \texttt{Monad}-Stacks. Hierzu schreiben Sie ein (sehr!) simples Spiel:\\
Durch drücken von \texttt{u} (up) bzw. \texttt{d} (down) wird ein interner Counter hoch- bzw. runtergezählt. Arbeiten Sie sich in den gegebenen Code (\texttt{game.hs}) ein und erstellen Sie die Game-Loop
\begin{minted}{haskell}
mainLoop :: RWST Env () State IO ()
\end{minted}
und die Tasteneingabe
\begin{minted}{haskell}
getInput :: RWST Env () State IO Input
\end{minted}
Benutzen Sie hierzu die gegebene \emph{pure} Hilfsfunktion
\begin{minted}{haskell}
getInputfromEnv :: Char -> Env -> Input
\end{minted}
\section*{Aufgabe \theuebungszettel.4:}
Erweitern Sie Ihr Spiel sinnvoll durch einen weiteren Zähler, der durch die Tasten \texttt{r} und \texttt{l} für rechts und links erhöht bzw. ernidriegt wird.
\end{document}