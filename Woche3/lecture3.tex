\documentclass{beamer}

\usepackage[utf8]{inputenc}
\usepackage[T1]{fontenc}
\usepackage[ngerman]{babel}
\usepackage{graphicx} % Bilder
\usepackage{wrapfig} % Umflussbilder
\usepackage{multicol} % Multiple columns
\usepackage{minted} % Haskell source code
\usepackage{framed} % Frames around source code
\usepackage[framemethod=tikz]{mdframed} % Frames
\usepackage{verbatim} % \begin{comment}...\end{comment}
\usepackage{etoolbox} % manipulate minted
\AtBeginEnvironment{minted}{\fontsize{10}{10}\selectfont}
\AfterEndEnvironment{minted}{}

\mdfdefinestyle{fancy}{
  roundcorner=5pt,
  linewidth=4pt,
  linecolor=red!80,
  backgroundcolor=red!20
}
\newmdenv[style=fancy]{important}

% redifine \em for \emph to use bold instead of italics
\makeatletter
\DeclareRobustCommand{\em}{%
  \@nomath\em \if b\expandafter\@car\f@series\@nil
  \normalfont \else \bfseries \fi}
\makeatother

% Stuff for Beamer
\beamertemplatenavigationsymbolsempty
\usetheme{Warsaw}

\title{Intermediate Dabbling in Haskell}

\begin{document}

%  \usebackgroundtemplate{\includegraphics[width=\paperwidth,height=\paperheight]{1.jpg}} 
  
%----------------------------------------------------------------------------------------  

  \begin{frame}
  \begin{center}
    \Huge\textbf{Intermediate Functional Programming in Haskell}\\ \bigskip
    \LARGE Universität Bielefeld, Sommersemester 2015\\ \bigskip
    \large Jonas Betzendahl \& Stefan Dresselhaus
    \end{center}
  \end{frame}

%----------------------------------------------------------------------------------------  
\begin{frame}[allowframebreaks]{Outline}
\frametitle{Übersicht}
\tableofcontents[hideallsubsections]
\end{frame}

\section{State-Monad}

\subsection{Beispiele}
\begin{frame}[fragile]
Wir hatten in der letzten Vorlesung die State-Monade kurz angesprochen.\\
Heute wenden wir uns der Definition zu und werden herausfinden, wie man noch weiter abstrahieren kann.\\
\end{frame}

\begin{frame}[fragile]
Beispiel:
\begin{minted}{haskell}
countme :: a -> State Int a
countme a = do
              modify (+1)
              return a

example :: State Int Int
example = do
              x <- countme (2+2)
              y <- return (x*x)
              z <- countme (y-2)
              return z

examplemain = runState example 0
-- -> (14,2), 14 = wert von z, 2 = interner counter
\end{minted}
\end{frame}

\begin{frame}[fragile]
Beispiel 2:
\begin{minted}{haskell}
module Main where
import Control.Monad.State
type CountValue = Int
type CountState = (Bool, Int)
 
startState :: CountState
startState = (False, 0)

play :: String -> State CountState CountValue
--play  ...

\end{minted}
\end{frame}

\begin{frame}[fragile]
\begin{minted}{haskell}
play []     = do
              (_, score) <- get
              return score
play (x:xs) = do
 (on, score) <- get
 case x of
   'C' -> if on then put (on, score + 1) else put (on, score)
   'A' -> if on then put (on, score - 1) else put (on, score)
   'T' -> put (False, score)
   'G' -> put (True, score)
   _   -> put (on, score)
 playGame xs

main = print $ runState (play "GACAACTCGAAT") startState
-- -> (-3,(False,-3))
\end{minted}
\end{frame}

\subsection{Definition}
\begin{frame}[fragile]
Definition von State:
\begin{minted}{haskell}
newtype State s a = State { runState :: s -> (a,s) }
\end{minted}
\pause
Diese (Record-)Notation liefert uns 2 Funktionen:
\begin{overprint}
\onslide<2>
\begin{minted}{haskell}
State    :: (s -> (a,s)) -> State s a
runState :: State s a    -> (s -> (a,s))
\end{minted}
\onslide<3->
\begin{minted}{haskell}
State    :: (s -> (a,s)) -> State s a
runState :: State s a    ->  s -> (a,s)
\end{minted}
\end{overprint}
\onslide<4->
\texttt{runState} benötigt also 2 Argumente, damit es ein \texttt{(a,s)} liefert.\bigskip \\
\onslide<5->
Wenn wir State monadisch nutzen, benutzen wir Funktionen der Form:
\begin{overprint}
\onslide<6>
\begin{minted}{haskell}
foo :: a -> State s b
\end{minted}
\onslide<7>
\begin{minted}{haskell}
foo :: a -> (s -> (b,s))
\end{minted}
\onslide<8->
\begin{minted}{haskell}
foo :: a -> s -> (b,s)
\end{minted}
\end{overprint}
\onslide<9->
State in der monadischen Form fügt einfach nur einen Funktionsparameter \texttt{s} hinzu und versteckt das \texttt{(b,s)} und gibt lediglich das \texttt{b} in der do-Notation zurück.
\end{frame}

\subsection{Functor/Applicative/Monad}
\begin{frame}[fragile]
Hilfreich ist es, sich die State-Monade als Berechnung vorzustellen, die noch nicht ausgeführt werden kann, weil der \textbf{initiale} State noch nicht bekannt ist.\bigskip \\
\pause
Man \textbf{bekommt} also erst einen State, bearbeitet ihn ggf. und gibt dann den geänderten State weiter.\\
\pause
Dies spiegelt sich auch in der Funktor-Instanz wieder:
\end{frame}

\begin{frame}[fragile]
\begin{overprint}
\onslide<1-2>
\begin{minted}{haskell}
instance Functor (State s) where
  fmap f rs = _

\end{minted}
\onslide<3>
\begin{minted}{haskell}
instance Functor (State s) where
  fmap f rs = State $ _

\end{minted}
\onslide<4-5>
\begin{minted}{haskell}
instance Functor (State s) where
  fmap f rs = State $ \s -> _

\end{minted}
\onslide<6>
\begin{minted}{haskell}
instance Functor (State s) where
  fmap f rs = State $ \s -> let (a,s') = runState rs s
                            in _
\end{minted}
\onslide<7>
\begin{minted}{haskell}
instance Functor (State s) where
  fmap f rs = State $ \s -> let (a,s') = runState rs s
                            in (f a, _)
\end{minted}
\onslide<8>
\begin{minted}{haskell}
instance Functor (State s) where
  fmap f rs = State $ \s -> let (a,s') = runState rs s
                            in (f a, s')
\end{minted}
\end{overprint}
\bigskip
\scriptsize
\begin{overprint}
\onslide<1>
\begin{verbatim}
Found hole ‘_’ with type: State s b
Where: ‘s’ is a rigid type variable
       ‘b’ is a rigid type variable
Relevant bindings include
  rs :: State s a
  f :: a -> b
  fmap :: (a -> b) -> State s a -> State s b
\end{verbatim}

\onslide<2>
\begin{minted}{haskell}
State :: (s -> (b,s)) -> State s b
\end{minted}
\onslide<3>
\begin{verbatim}
Found hole ‘_’ with type: s -> (b, s)
Where: ‘s’ is a rigid type variable
       ‘b’ is a rigid type variable
Relevant bindings include
  rs :: State s a
  f :: a -> b
  fmap :: (a -> b) -> State s a -> State s b
\end{verbatim}
\onslide<4>
\begin{verbatim}
Found hole ‘_’ with type: (b, s)
Where: ‘s’ is a rigid type variable
       ‘b’ is a rigid type variable
Relevant bindings include
  s :: s
  rs :: State s a
  f :: a -> b
  fmap :: (a -> b) -> State s a -> State s b
\end{verbatim}
\onslide<5>
\begin{verbatim}
Found hole ‘_’ with type: (b, s)
Where: ‘s’ is a rigid type variable
       ‘b’ is a rigid type variable
Relevant bindings include
  s :: s
  rs :: State s a
  f :: a -> b
  fmap :: (a -> b) -> State s a -> State s b
\end{verbatim}
\begin{minted}{haskell}
runState :: State s a -> s -> (a,s)
\end{minted}

\onslide<6>
\begin{verbatim}
Found hole ‘_’ with type: (b, s)
Where: ‘s’ is a rigid type variable
       ‘b’ is a rigid type variable
Relevant bindings include
  a :: a
  s' :: s
  s :: s
  rs :: State s a
  f :: a -> b
  fmap :: (a -> b) -> State s a -> State s b
\end{verbatim}
\onslide<7>
\begin{verbatim}
Found hole ‘_’ with type: s
Where: ‘s’ is a rigid type variable
Relevant bindings include
  a :: a
  s' :: s
  s :: s
  rs :: State s a
  f :: a -> b
  fmap :: (a -> b) -> State s a -> State s b
\end{verbatim}
\end{overprint}
\normalsize
\bigskip
\onslide<8>
Danke, typed holes!
\end{frame}

\begin{frame}[fragile]
Ganz analog funktioniert die Applicative-Instanz:
\begin{overprint}
\onslide<2>
\begin{minted}{haskell}
instance Applicative (State s) where
    pure a    = _
    rf <*> rs = undefined

    
    
\end{minted}
\onslide<3>
\begin{minted}{haskell}
instance Applicative (State s) where
    pure a    = State $ \s -> _
    rf <*> rs = undefined

    
    
\end{minted}
\onslide<4>
\begin{minted}{haskell}
instance Applicative (State s) where
    pure a    = State $ \s -> (a,s)
    rf <*> rs = State $ \s -> _

    
    
\end{minted}
\onslide<5-6>
\begin{minted}{haskell}
instance Applicative (State s) where
    pure a    = State $ \s -> (a,s)
    rf <*> rs = State $ \s ->
                  let (f,s')  = runState rf s
                      (a,s'') = runState rs s'
                  in _
\end{minted}
\onslide<7>
\begin{minted}{haskell}
instance Applicative (State s) where
    pure a    = State $ \s -> (a,s)
    rf <*> rs = State $ \s ->
                  let (f,s')  = runState rf s
                      (a,s'') = runState rs s'
                  in (f a, s'')
\end{minted}
\end{overprint}
\bigskip
\scriptsize
\begin{overprint}
\onslide<2>
\begin{verbatim}
Found hole ‘_’ with type: State s a
Where: ‘s’ is a rigid type variable
       ‘a’ is a rigid type variable
Relevant bindings include
  a :: a
  pure :: a -> State s a
\end{verbatim}
\onslide<3>
\begin{verbatim}
Found hole ‘_’ with type: (a, s)
Where: ‘s’ is a rigid type variable
       ‘a’ is a rigid type variable
Relevant bindings include
  s :: s
  a :: a
  pure :: a -> State s a
\end{verbatim}
\onslide<4>
\begin{verbatim}
Found hole ‘_’ with type: (b, s)
Where: ‘s’ is a rigid type variable
       ‘b’ is a rigid type variable
Relevant bindings include
  s :: s
  rs :: State s a
  rf :: State s (a -> b)
  (<*>) :: State s (a -> b) -> State s a -> State s b
\end{verbatim}
\onslide<5>
\normalsize
Wichtig: Erst das \texttt{rf} ausführen, dann das \texttt{rs}, da \texttt{<*>} von rechts-nach-links arbeitet.
\onslide<6>
\scriptsize
\begin{verbatim}
Found hole ‘_’ with type: (b, s)
Where: ‘s’ is a rigid type variable
       ‘b’ is a rigid type variable
Relevant bindings include
  a :: a
  s'' :: s
  f :: a -> b
  s' :: s
  s :: s
  rs :: State s a
  rf :: State s (a -> b)
  (<*>) :: State s (a -> b) -> State s a -> State s b
\end{verbatim}
\end{overprint}
\end{frame}

\begin{frame}[fragile]
Am wichtigsten ist die Monad-Instanz:
\begin{overprint}
\onslide<2>
\begin{minted}{haskell}
instance Monad (State s) where
    return   = pure
    rs >>= f = State $ \s -> _

    
    
\end{minted}
\onslide<3>
\begin{minted}{haskell}
instance Monad (State s) where
    return   = pure
    rs >>= f = State $ \s ->
                let (a,s') = runState rs s
                in _
    
\end{minted}
\onslide<4>
\begin{minted}{haskell}
instance Monad (State s) where
    return   = pure
    rs >>= f = State $ \s ->
                let (a,s') = runState rs s
                    rs'    = f a
                in _
\end{minted}
\onslide<5>
\begin{minted}{haskell}
instance Monad (State s) where
    return   = pure
    rs >>= f = State $ \s ->
                let (a,s') = runState rs s
                    rs'    = f a
                in runState rs' s'
\end{minted}
\end{overprint}
\bigskip
\scriptsize
\begin{overprint}
\onslide<2>
\begin{verbatim}
Found hole ‘_’ with type: (b, s)
Where: ‘s’ is a rigid type variable
       ‘b’ is a rigid type variable
Relevant bindings include
  s :: s
  f :: a -> State s b
  rs :: State s a
  (>>=) :: State s a -> (a -> State s b) -> State s b
\end{verbatim}
\onslide<3>
\begin{verbatim}
Found hole ‘_’ with type: (b, s)
Where: ‘s’ is a rigid type variable
       ‘b’ is a rigid type variable
Relevant bindings include
  a :: a
  s' :: s
  s :: s
  f :: a -> State s b
  rs :: State s a
  (>>=) :: State s a -> (a -> State s b) -> State s b
\end{verbatim}
\onslide<4>
\begin{verbatim}
Found hole ‘_’ with type: (b, s)
Where: ‘s’ is a rigid type variable
       ‘b’ is a rigid type variable
Relevant bindings include
  rs' :: State s b
  a :: a
  s' :: s
  s :: s
  f :: a -> State s b
  rs :: State s a
  ...
\end{verbatim}
\end{overprint}
\end{frame}

\section{Monad-Transformer}

\subsection{Beispiel}
\begin{frame}[fragile]
Wir hatten letzte Woche die Maybe-Monade mit dem folgenden Anwendugsfall:
\begin{minted}{haskell}
f = do
      folder <- getInbox
      mail   <- getFirstMail folder
      header <- getHeader mail
      return header
\end{minted}
\pause
Nun möchten wir aus irgendeinem Grund (Logging, Netzwerk, ..) zwischen dem \texttt{getInbox} und dem \texttt{getFirstMail} eine IO-Aktion ausführen.\\
\bigskip \pause
Problem: \texttt{IO /= Maybe}\\
\pause
Als Konsequenz können wir die do-notation nicht verwenden - wir fallen also wieder zurück auf die hässliche Notation:
\end{frame}

\begin{frame}[fragile]
\begin{minted}{haskell}
f :: IO (Maybe Header)
f = case getInbox of
     (Just folder) -> 
        do
          putStrLn "debug"
          case getFirstMail folder of
            (Just mail) -> 
               case getHeader mail of
                 (Just head) -> return $ return head
                 Nothing     -> return Nothing
            Nothing     -> return Nothing
     Nothing       -> return Nothing
\end{minted}
\end{frame}

\begin{frame}[fragile]
Dieser Code ist ohne Frage hässlich. Stellt sich die Frage, ob wir nicht soetwas, wie \texttt{MaybeIO} bauen können, sodass wir wieder do-notation verwenden können.\\
\pause
Also kombinieren wir es (ähnlich zur State-Monade):
\begin{minted}{haskell}
newtype MaybeIO a = MaybeIO { runMaybeIO :: IO (Maybe a) }
\end{minted}
\pause
Dieses liefert uns 2 Funktionen:
\begin{minted}{haskell}
MaybeIO    :: IO (Maybe a) -> MaybeIO a
runMaybeIO :: MaybeIO a -> IO (Maybe a)
\end{minted}
Also eine Funktion, um in unsere neue Monade zu kommen und eine Funktion um dieses wieder Rückgängig zu machen.
\end{frame}

\subsection{F/A/M}
\begin{frame}[fragile]
Fangen wir mit der Functor-Instanz an:
\begin{overprint}
\onslide<2>
\begin{minted}{haskell}
instance Functor MaybeIO where
  fmap f input = _

  
  
                 
\end{minted}
\onslide<3>
\begin{minted}{haskell}
instance Functor MaybeIO where
  fmap f input = _
               where
                 unwrapped = runMaybeIO input

                 
\end{minted}
\onslide<4>
\begin{minted}{haskell}
instance Functor MaybeIO where
  fmap f input = _
               where
                 unwrapped = runMaybeIO input
                 fmapped = fmap (fmap f) unwrapped

\end{minted}
\onslide<5>
\begin{minted}{haskell}
instance Functor MaybeIO where
  fmap f input = _
               where
                 unwrapped = runMaybeIO input
                 fmapped = fmap (fmap f) unwrapped
                 wrapped = MaybeIO fmapped
\end{minted}
\onslide<6-7>
\begin{minted}{haskell}
instance Functor MaybeIO where
  fmap f input = wrapped
               where
                 unwrapped = runMaybeIO input
                 fmapped = fmap (fmap f) unwrapped
                 wrapped = MaybeIO fmapped
\end{minted}
\end{overprint}
\bigskip
\scriptsize
\begin{overprint}
\onslide<2>
\begin{verbatim}
Found hole ‘_’ with type: MaybeIO b
Where: ‘b’ is a rigid type variable
Relevant bindings include
  input :: MaybeIO a
  f :: a -> b
  fmap :: (a -> b) -> MaybeIO a -> MaybeIO b
\end{verbatim}
\onslide<3>
\begin{verbatim}
Found hole ‘_’ with type: MaybeIO b
Where: ‘b’ is a rigid type variable
Relevant bindings include
  unwrapped :: IO (Maybe a)
  input :: MaybeIO a
  f :: a -> b
  fmap :: (a -> b) -> MaybeIO a -> MaybeIO b
\end{verbatim}
\onslide<4>
\begin{verbatim}
Found hole ‘_’ with type: MaybeIO b
Where: ‘b’ is a rigid type variable
Relevant bindings include
  fmapped :: IO (Maybe b)
  unwrapped :: IO (Maybe a)
  input :: MaybeIO a
  f :: a -> b
  fmap :: (a -> b) -> MaybeIO a -> MaybeIO b
\end{verbatim}
\onslide<5>
\begin{verbatim}
Found hole ‘_’ with type: MaybeIO b
Where: ‘b’ is a rigid type variable
Relevant bindings include
  wrapped :: MaybeIO b
  fmapped :: IO (Maybe b)
  unwrapped :: IO (Maybe a)
  input :: MaybeIO a
  f :: a -> b
  fmap :: (a -> b) -> MaybeIO a -> MaybeIO b
\end{verbatim}
\normalsize
\onslide<7>
oder kurz:
\begin{minted}{haskell}
instance Functor MaybeIO where
  fmap f = MaybeIO . fmap (fmap f) . runMaybeIO
\end{minted}
\end{overprint}
\end{frame}

\begin{frame}[fragile]
Applicative:
\begin{minted}{haskell}
instance Applicative MaybeIO where
  pure    = MaybeIO . pure . Just
            -- in Just packen, mit pure in IO heben
            -- und den Typen mit MaybeIO aliasen
  f <*> x = MaybeIO $ (<*>) <$> f' <*> x'
            where
              f' = runMaybeIO f -- IO (Maybe f)
              x' = runMaybeIO x -- IO (Maybe x)
\end{minted}
Das erste \texttt{(<*>)} ist Applicative auf Maybe und es wird in Applicative \texttt{<*>} von IO hineingemappt.
\end{frame}

\begin{frame}[fragile]
Monad:
\begin{minted}{haskell}
instance Monad MaybeIO where
  return = pure
  x >>= f = MaybeIO $ x''
            where
              x' = runMaybeIO x
              x'' = x' >>= runMaybeIO . mb . fmap f
              mb :: Maybe (MaybeIO a) -> MaybeIO a
              mb (Just a) = a
              mb Nothing = MaybeIO $ return Nothing
\end{minted}
Zuerst packen wir das MaybeIO aus. \texttt{fmap f} bringt uns ein Maybe (MaybeIO a), welches wir mittels der Hilfsfunktion \texttt{mb} auspacken oder einen leeren Wert konstruieren.\\
Dieses jagen wir noch durch runMaybeIO um wieder ein \texttt{IO (Maybe a)} zu bekommen, auf das wir dann den \texttt{>==}-Operator von IO anwenden können. Das Ergebnis verpacken wir noch in \texttt{MaybeIO} und sind fertig.
\end{frame}

\subsection{Beispiel revisited}
\begin{frame}[fragile]
Da wir nun eine Monade definiert haben, können wir ja wieder do nutzen:
\begin{minted}{haskell}
f = do
     i <- getInbox
     putStrLn "debug"
     m <- getFirstMail i
     h <- getHeader m
     return h
\end{minted}
\end{frame}
\begin{frame}[fragile]
Allerdings:
\begin{minted}{text}
    Couldn't match type Maybe with MaybeIO
    Expected type: MaybeIO Inbox
      Actual type: Maybe Inbox
    In a stmt of a 'do' block: in <- getInbox

    Couldn't match type IO with MaybeIO
    Expected type: MaybeIO ()
      Actual type: IO ()
    In a stmt of a 'do' block: putStrLn "debug"

    Couldn't match type Maybe with MaybeIO
    Expected type: MaybeIO Mail
      Actual type: Maybe Mail
    In a stmt of a 'do' block: m <- getFirstMail i
    
    Couldn't match type Maybe with MaybeIO
    Expected type: MaybeIO Header
      Actual type: Maybe Header
    In a stmt of a 'do' block: h <- getHeader m
\end{minted}
\end{frame}

\begin{frame}[fragile]
Wir brauchen also Konverter:
\begin{itemize}
 \item \texttt{Maybe -> MaybeIO}
 \item \texttt{IO -> MaybeIO}
\end{itemize}
\pause
Aber wir haben schon alles, was wir brauchen, wenn wir uns nur klar machen:
\begin{minted}{haskell}
return  :: Maybe a -> IO (Maybe a) -- return von IO
MaybeIO :: IO (Maybe a) -> MaybeIO a
\end{minted}
\pause
und
\begin{minted}{haskell}
Just      :: a -> Maybe a
fmap Just :: IO a -> IO (Maybe a)
\end{minted}
\end{frame}

\begin{frame}[fragile]
Somit wird unser Code von oben:
\begin{minted}{haskell}
f = do
     i <- MaybeIO (return (getInbox))
     MaybeIO (fmap Just (putStrLn "debug"))
     m <- MaybeIO (return (getFirstMail i))
     h <- MaybeIO (return (getHeader m))
     return h
\end{minted}
\pause
Zwar können wir nun do nutzen, aber das sieht doch eher hässlich aus. Außerdem ist so viel Code doppelt!
\end{frame}

\subsection{Finale Version}

\begin{frame}[fragile]
Wenn wir Muster finden, dann faktorisieren wir sie doch raus!
\begin{minted}{haskell}
liftMaybe :: Maybe a -> MaybeIO a
liftMaybe x = MaybeIO (return x)

liftIO :: IO a -> MaybeIO a
liftIO x = MaybeIO (fmap Just x)
\end{minted}
\pause
und wir erhalten:
\begin{minted}{haskell}
f = do
     i <- liftMaybe getInbox
     liftIO $ putStrLn "debug"
     m <- liftMaybe $ getFirstMail i
     h <- liftMaybe $ getHeader m
     return h
\end{minted}
\end{frame}

\section{Monad-Transformer cont.}

\subsection{Recap}

\begin{frame}[fragile]
Wenn wir uns nochmals ansehen, welche Eigenschaft der IO-Monade wir genutzt haben, dann fällt uns auf:
\pause
\begin{minted}{haskell}
instance Functor MaybeIO where
  fmap f = MaybeIO . fmap (fmap f) . runMaybeIO
\end{minted}
\texttt{fmap} von IO als Funktor
\pause
\begin{minted}{haskell}
instance Applicative MaybeIO where
  pure    = MaybeIO . pure . Just
  f <*> x = MaybeIO $ (<*>) <$> (runMaybeIO f)
                            <*> (runMaybeIO x)
\end{minted}
\texttt{pure} und \texttt{<*>} von IO als Applicative
\pause
\begin{minted}{haskell}
instance Monad MaybeIO where
  return = pure
  x >>= f = MaybeIO $ (runMaybeIO x)
                      >>= runMaybeIO . mb . fmap f
            where
              mb (Just a) = a
              mb Nothing = MaybeIO $ return Nothing
\end{minted}
\texttt{return} und \texttt{>>=} von IO
\end{frame}

\begin{frame}[fragile]
Uns fällt auf: Wir verwenden gar keine intrisischen Eigenschaften von IO.\\
Also können wir IO auch durch jede Monade ersetzten. Dies nennt man dann Monad Transformer.
\begin{minted}{haskell}
data MaybeT m a = MaybeT { runMaybeT :: m (Maybe a) }
\end{minted}
\end{frame}

\begin{frame}[fragile]
Und der Code von eben wird zu:
\begin{minted}{haskell}
instance Functor m => Functor (MaybeT m) where
  fmap f = MaybeT . fmap (fmap f) . runMaybeT

instance Applicative m => Applicative (MaybeT m) where
  pure    = MaybeT . pure . Just
  f <*> x = MaybeT $ (<*>) <$> (runMaybeT f)
                           <*> (runMaybeT x)

instance Monad m => Monad (MaybeT m) where
  return = pure
  x >>= f = MaybeT $ (runMaybeT x)
                     >>= runMaybeT . mb . fmap f
            where
              mb (Just a) = a
              mb Nothing = MaybeT $ return Nothing
\end{minted}
\pause
Wir haben nur die Monade heraus faktorisiert.
\end{frame}

\begin{frame}[fragile]
Frage: Wie realisieren wir nun \texttt{liftIO} etc.?\\
\pause
Über Typklassen!
\begin{minted}{haskell}
class Monad m => MonadIO m where
    liftIO :: IO a -> m a
\end{minted}
Wir verlangen einfach, dass IO irgendwie verarbeitet werden muss.
\pause
\begin{important}
IO ist ein Spezialfall, da IO als Monade nicht additiv ist. Es gibt somit keinen IOT.
\end{important}
\end{frame}

\begin{frame}[fragile]
Frage: Wie realisieren wir nun \texttt{liftIO} etc.?\\
Über Typklassen!
\begin{minted}{haskell}
class MonadTrans t where
    lift :: (Monad m) => m a -> t m a
\end{minted}
\pause
Dies ist die allgemeine Form für additive Monaden. Mit \texttt{lift} heben wir uns eine monadische Ebene höher.\\
Doch was bedeutet das?
\end{frame}

\subsection{Beispiele}

\begin{frame}[fragile]
Wir haben schon ein paar Monaden kennengelernt. Diese sind fast alle additiv. Wir können somit folgendes bauen:
\pause
\begin{minted}{haskell}
data MyMonadStack a = StateT (EitherT (MaybeT (IO a)))
\end{minted}
\pause
Wie schreiben wir nun hierin Code?
\begin{minted}{haskell}
bsp :: MyMonadStack ()
bsp = do
    a <- fun                -- fun  :: MyMonadStack Int
    b <- lift $ fun2        -- fun2 :: EitherT (MaybeT (IO Int))
    c <- lift . lift $ fun3 -- fun3 :: MaybeT (IO Int))
    liftIO $ putStrLn "foo"
\end{minted}

\end{frame}

\begin{frame}[fragile]
Um auf spezielle Ebenen im Monad-Stack zuzugreifen gibt es für jeden Zweck eine Typklasse. Beispielsweise:
\pause
\begin{minted}{haskell}
instance (Monad m) => MonadState s (StateT s m) where
  get   = StateT $ \s -> return (s,s)
  put s = StateT $ \_ -> return ((),s)
\end{minted}
\pause
Nun können wir einfach
\begin{minted}{haskell}
bsp :: MyMonadStack ()
bsp = do
    state <- get
    put $ manipulateState state
    liftIO $ putStrLn "foo"
\end{minted}
\pause
Analog gibt es dies auch für \texttt{ReaderT} (\texttt{env}), welches ein read-only-Environment bereitstellt (z.b. eine Konfiguration) oder für \texttt{WriterT} (\texttt{tell}), welches ein write-only-Environment zur Verfügung stellt (z.b. Logging). Häufig findet man daher einen Read-Write-State-Transformer, kurz \texttt{RWST}-Stack.\\
\pause
Echtweltprogramme sind meist durch eine \texttt{RWST IO} mit der Außenwelt verbunden.
\end{frame}

\begin{frame}[fragile]
Ein weiteres Echtwelt-Beispiel könnte etwa der folgende Aufruf sein:
\begin{minted}{haskell}
data Env = Env { filename :: String }

readInputs :: ReaderT Env IO String
readInputs = do
           e <- env
           f <- liftIO $ readFile (filename e)
           return f
\end{minted}
\pause
Dieser Aufruf liest einen Dateinamen aus einem Environment, kann per \texttt{liftIO} IO-Aktionen ausführen und das Ergebnis (den String mit dem Dateiinhalt) zurückliefern.
\end{frame}

\begin{frame}[fragile]
Noch ein Beispiel aus einem Spiel könnte sein:
\begin{minted}{haskell}
mainLoop :: RWST Env () State IO ()
mainLoop = do
         e <- env
         f <- liftIO $ getUserInput (keySettings e)
         newWorld <- updateWorld f 
                     --holt sich die Welt mittels 'get'
         put newWorld
         unless (f == endKey e) mainLoop
\end{minted}
\pause
Dies ist ein klassisches Game-Loop, bestehend aus Konfigurationen im Env (Key settings), IO (User-Input abfragen), Update des internen Zustands (updateWorld) und das schreiben des neuen Zustandes (put newWorld). Sofern dann das Spiel nicht beendet ist loopen wir.
\end{frame}



\end{document}