\documentclass{beamer}

\usepackage[utf8]{inputenc}
\usepackage[T1]{fontenc}
\usepackage[ngerman]{babel}
\usepackage{graphicx} % Bilder
\usepackage{wrapfig} % Umflussbilder
\usepackage{multicol} % Multiple columns
\usepackage{minted} % Haskell source code
\usepackage{framed} % Frames around source code
\usepackage[framemethod=tikz]{mdframed} % Frames
\usepackage{verbatim} % \begin{comment}...\end{comment}
\usepackage{etoolbox} % manipulate minted
\AtBeginEnvironment{minted}{\fontsize{10}{10}\selectfont}
\AfterEndEnvironment{minted}{}

\mdfdefinestyle{fancy}{
  roundcorner=5pt,
  linewidth=4pt,
  linecolor=red!80,
  backgroundcolor=red!20
}
\newmdenv[style=fancy]{important}

% redifine \em for \emph to use bold instead of italics
\makeatletter
\DeclareRobustCommand{\em}{%
  \@nomath\em \if b\expandafter\@car\f@series\@nil
  \normalfont \else \bfseries \fi}
\makeatother

% Stuff for Beamer
\beamertemplatenavigationsymbolsempty
\usetheme{Warsaw}

\title{Fortgeschrittene Funktionale Programmierung in Haskell}

\begin{document}

%  \usebackgroundtemplate{\includegraphics[width=\paperwidth,height=\paperheight]{1.jpg}} 
  
%----------------------------------------------------------------------------------------  

  \begin{frame}
  \begin{center}
    \huge\textbf{Fortgeschrittene Funktionale Programmierung in Haskell}\\ \bigskip
    \LARGE Universität Bielefeld, Sommersemester 2015\\ \bigskip
    \large Jonas Betzendahl \& Stefan Dresselhaus
    \end{center}
  \end{frame}

%----------------------------------------------------------------------------------------  
\begin{frame}[allowframebreaks]{Outline}
\frametitle{Übersicht}
\tableofcontents[hideallsubsections]
\end{frame}

\section{Record-Syntax}

\subsection{Beispiel}

\begin{frame}[fragile]
Nehmen wir an, wir wollen den folgenden Produkttypen definieren:
\begin{minted}{haskell}
data V2 a = V2 a a
\end{minted}
\pause
Nun wollen wir auch einzelne Elemente auslesen und setzen (sprich: Eine neue Datenstruktur, identisch zur alten, mit dem \glqq gesetzten\grqq\ Wert als einzigem Unterschied). \smallskip

Dafür müssen wir ein paar Hilfsfunktionen definieren.
\end{frame}

\begin{frame}[fragile]
Zum Auslesen:
\pause
\begin{minted}{haskell}
x :: V2 a -> a
x (V2 x' _) = x'
\end{minted}
\pause
\begin{minted}{haskell}
y :: V2 a -> a
y (V2 _ y') = y'
\end{minted}
\pause
\bigskip

Und zum Setzen:
\pause
\begin{minted}{haskell}
sx :: a -> V2 a -> V2 a
sx x (V2 _ y) = V2 x y
\end{minted}
\pause
\begin{minted}{haskell}
sy :: a -> V2 a -> V2 a
sy y (V2 x _) = V2 x y
\end{minted}
\end{frame}

\subsection{Definition}

\begin{frame}[fragile]
Nun sind diese Funktionen stupide zu schreiben. Man kann diesen Vorgang allerdings auch direkt bei der Deklaration des Typen automatisieren.\\\par
\pause
Dies nennt sich dann \emph{Record-Syntax}:\smallskip

\pause
\begin{minted}{haskell}
data V2 a = V2 { x :: a
               , y :: a
               }
\end{minted}
\pause
welches automatisch die Funktionen
\begin{minted}{haskell}
x :: V2 a -> a
y :: V2 a -> a
\end{minted}
generiert.
\end{frame}

\begin{frame}[fragile]
Es werden auch Setter generiert, die wie folgt zu verwenden sind: \bigskip
\pause
\begin{minted}{haskell}
let r = V2 1 2
let r' = r { x = 0 }
-- r' = V2 0 2
\end{minted}
\pause
\bigskip

Wir geben also einfach die Struktur an (hier: \texttt{r}) und in den geschweiften Klammern alle Parameter, die wir ändern wollen.
\end{frame}

\subsection{Ein weiteres Beispiel}

\begin{frame}[fragile]
Nochmal ein etwas komplizierteres Beispiel:\\\par
\pause
Welche Funktionen generiert folgender Code? \smallskip

\begin{minted}{haskell}
data D a = K a (a -> a)
\end{minted}
\pause
\begin{minted}{haskell}
K :: a -> (a -> a) -> D a
\end{minted}
\pause
Welche Funktionen generiert folgende Record-Syntax?
\begin{minted}{haskell}
data D a = K { x :: a, y :: (a -> a) }
\end{minted}
\pause
\begin{minted}{haskell}
K :: a -> (a -> a) -> D a
x :: D a -> a
y :: D a -> (a -> a)
\end{minted}
\pause
Die Record-Syntax beschert uns also kostenlos Getter-Funktionen und bietet die Möglichkeit eines Setzens über die Update-Notation \texttt{a \{ x = y \}}
\end{frame}

\section{State-Monad}

\subsection{Beispiele}
\begin{frame}[fragile]
Wir hatten in der letzten Vorlesung die \texttt{State}-Monade kurz angesprochen.\\
Heute wenden wir uns der Definition zu und werden herausfinden, wie man noch weiter abstrahieren kann.\\
\end{frame}

\begin{frame}[fragile]
Beispiel:\bigskip

\begin{minted}{haskell}
countme :: a -> State Int a
countme a = do modify (+1)
               return a

example :: State Int Int
example = do x <- countme (2+2)
             y <- return (x*x)
             z <- countme (y-2)
             return z

examplemain = runState example 0
-- -> (14,2), 14 ist der Wert von z, 2 ist der interner Counter
\end{minted}
\end{frame}

\begin{frame}[fragile]
Beispiel 2:\bigskip

\begin{minted}{haskell}
module Main where

import Control.Monad.State

type CountValue = Int
type CountState = (Bool, Int)
 
startState :: CountState
startState = (False, 0)

play :: String -> State CountState CountValue
--play  ...

\end{minted}
\end{frame}

\begin{frame}[fragile]
\begin{minted}{haskell}
play :: String -> State CountState CountValue
play []     = do (_, score) <- get
                 return score
play (x:xs) = do
 (on, score) <- get
 case x of
   'C' -> if on then put (on, score + 1) else put (on, score)
   'A' -> if on then put (on, score - 1) else put (on, score)
   'T' -> put (False, score)
   'G' -> put (True, score)
   _   -> put (on, score)
 play xs

main :: IO ()
main = print $ runState (play "GACAACTCGAAT") startState
-- -> (-3,(False,-3))
\end{minted}
%$
\end{frame}

\subsection{Definition}
\begin{frame}[fragile]
Definition von \texttt{State}:
\begin{minted}{haskell}
newtype State s a = State { runState :: s -> (a,s) }
\end{minted}
\pause
Diese (Record-)Notation liefert uns zwei Funktionen:
\begin{overprint}
\onslide<2>
\begin{minted}{haskell}
State    :: (s -> (a,s)) -> State s a
runState :: State s a    -> (s -> (a,s))
\end{minted}
\onslide<3->
\begin{minted}{haskell}
State    :: (s -> (a,s)) -> State s a
runState :: State s a    ->  s -> (a,s)
\end{minted}
\end{overprint}
\onslide<4->
\smallskip

\texttt{runState} benötigt also zwei Argumente, damit es ein \texttt{(a,s)} liefert.\bigskip \\
\onslide<5->
Wenn wir State monadisch nutzen, benutzen wir Funktionen der folgenden Form:
\begin{overprint}
\onslide<6>
\begin{minted}{haskell}
foo :: a -> State s b
\end{minted}
\onslide<7>
\begin{minted}{haskell}
foo :: a -> (s -> (b,s))
\end{minted}
\onslide<8->
\begin{minted}{haskell}
foo :: a -> s -> (b,s)
\end{minted}
\end{overprint}
\onslide<9->
State in der monadischen Form fügt einfach nur einen Funktionsparameter \texttt{s} hinzu, versteckt das \texttt{(b,s)} und gibt lediglich das \texttt{b} in der \texttt{do}-Notation zurück.
\end{frame}

\subsection{Functor/Applicative/Monad}
\begin{frame}[fragile]
Hilfreich ist es, sich die \texttt{State}-Monade als Berechnung vorzustellen, die noch nicht ausgeführt werden kann, weil der \textbf{Anfangzustand} (initial State) noch nicht bekannt ist.\bigskip

\pause
Man bekommt also erst \emph{später} irgendwann einen State, bearbeitet ihn ggf. und gibt dann den geänderten State weiter.\bigskip

\pause
Dies spiegelt sich auch in der \texttt{Functor}-Instanz wieder:
\end{frame}

\begin{frame}[fragile]
\begin{overprint}
\onslide<1-2>
\begin{minted}{haskell}
instance Functor (State s) where
  fmap f rs = _

\end{minted}
\onslide<3>
\begin{minted}{haskell}
instance Functor (State s) where
  fmap f rs = State $ _

\end{minted}
\onslide<4-5>
\begin{minted}{haskell}
instance Functor (State s) where
  fmap f rs = State $ \st -> _

\end{minted}
\onslide<6>
\begin{minted}{haskell}
instance Functor (State s) where
  fmap f rs = State $ \st -> let (a,st') = runState rs st
                             in _
\end{minted}
\onslide<7>
\begin{minted}{haskell}
instance Functor (State s) where
  fmap f rs = State $ \st -> let (a,st') = runState rs st
                             in (f a, _)
\end{minted}
\onslide<8>
\begin{minted}{haskell}
instance Functor (State s) where
  fmap f rs = State $ \st -> let (a,st') = runState rs st
                             in (f a, st')
\end{minted}
\end{overprint}
\bigskip
\scriptsize
\begin{overprint}
\onslide<1>
\begin{verbatim}
Found hole ‘_’ with type: State s b
Where: ‘s’ is a rigid type variable
       ‘b’ is a rigid type variable
Relevant bindings include
  rs :: State s a
  f :: a -> b
  fmap :: (a -> b) -> State s a -> State s b
\end{verbatim}

\onslide<2>
\begin{minted}{haskell}
State :: (s -> (b,s)) -> State s b
\end{minted}
\onslide<3>
\begin{verbatim}
Found hole ‘_’ with type: s -> (b, s)
Where: ‘s’ is a rigid type variable
       ‘b’ is a rigid type variable
Relevant bindings include
  rs :: State s a
  f :: a -> b
  fmap :: (a -> b) -> State s a -> State s b
\end{verbatim}
\onslide<4>
\begin{verbatim}
Found hole ‘_’ with type: (b, s)
Where: ‘s’ is a rigid type variable
       ‘b’ is a rigid type variable
Relevant bindings include
  st :: s
  rs :: State s a
  f :: a -> b
  fmap :: (a -> b) -> State s a -> State s b
\end{verbatim}
\onslide<5>
\begin{verbatim}
Found hole ‘_’ with type: (b, s)
Where: ‘s’ is a rigid type variable
       ‘b’ is a rigid type variable
Relevant bindings include
  st :: s
  rs :: State s a
  f :: a -> b
  fmap :: (a -> b) -> State s a -> State s b
\end{verbatim}
\begin{minted}{haskell}
runState :: State s a -> s -> (a,s)
\end{minted}

\onslide<6>
\begin{verbatim}
Found hole ‘_’ with type: (b, s)
Where: ‘s’ is a rigid type variable
       ‘b’ is a rigid type variable
Relevant bindings include
  a :: a
  st' :: s
  st :: s
  rs :: State s a
  f :: a -> b
  fmap :: (a -> b) -> State s a -> State s b
\end{verbatim}
\onslide<7>
\begin{verbatim}
Found hole ‘_’ with type: s
Where: ‘s’ is a rigid type variable
Relevant bindings include
  a :: a
  st' :: s
  st :: s
  rs :: State s a
  f :: a -> b
  fmap :: (a -> b) -> State s a -> State s b
\end{verbatim}
\end{overprint}
\normalsize
\bigskip
\onslide<8>
Danke, typed holes!
\end{frame}

\begin{frame}[fragile]
Ganz analog funktioniert die \texttt{Applicative}-Instanz:
\begin{overprint}
\onslide<2>
\begin{minted}{haskell}
instance Applicative (State s) where
    pure a    = _
    rf <*> rs = undefined

    
    
\end{minted}
\onslide<3>
\begin{minted}{haskell}
instance Applicative (State s) where
    pure a    = State $ _
    rf <*> rs = undefined

    
    
\end{minted}
\onslide<4>
\begin{minted}{haskell}
instance Applicative (State s) where
    pure a    = State $ \st -> _
    rf <*> rs = undefined

    
    
\end{minted}
\onslide<5>
\begin{minted}{haskell}
instance Applicative (State s) where
    pure a    = State $ \st -> (a,st)
    rf <*> rs = State $ \st -> _

    
    
\end{minted}
\onslide<6-7>
\begin{minted}{haskell}
instance Applicative (State s) where
    pure a    = State $ \st -> (a,st)
    rf <*> rs = State $ \st ->
                  let (f,st')  = runState rf st
                      (a,st'') = runState rs st'
                  in _
\end{minted}
\onslide<8>
\begin{minted}{haskell}
instance Applicative (State s) where
    pure a    = State $ \st -> (a,st)
    rf <*> rs = State $ \st ->
                  let (f,st')  = runState rf st
                      (a,st'') = runState rs st'
                  in (f a, st'')
\end{minted}
\end{overprint}
\bigskip
\scriptsize
\begin{overprint}
\onslide<2>
\begin{verbatim}
Found hole ‘_’ with type: State s a
Where: ‘s’ is a rigid type variable
       ‘a’ is a rigid type variable
Relevant bindings include
  a :: a
  pure :: a -> State s a
\end{verbatim}
\onslide<3>
\begin{verbatim}
Found hole ‘_’ with type: s -> (a, s)
Where: ‘s’ is a rigid type variable
       ‘a’ is a rigid type variable
Relevant bindings include
  a :: a
  pure :: a -> State s a
\end{verbatim}
\onslide<4>
\begin{verbatim}
Found hole ‘_’ with type: (a, s)
Where: ‘s’ is a rigid type variable
       ‘a’ is a rigid type variable
Relevant bindings include
  st :: s
  a :: a
  pure :: a -> State s a
\end{verbatim}
\onslide<5>
\begin{verbatim}
Found hole ‘_’ with type: (b, s)
Where: ‘s’ is a rigid type variable
       ‘b’ is a rigid type variable
Relevant bindings include
  st :: s
  rs :: State s a
  rf :: State s (a -> b)
  (<*>) :: State s (a -> b) -> State s a -> State s b
\end{verbatim}
\onslide<6>
\normalsize
Wichtig: Erst das \texttt{rf} ausführen, dann das \texttt{rs}, da \texttt{<*>} von links-nach-rechts arbeitet.
\onslide<7>
\scriptsize
\begin{verbatim}
Found hole ‘_’ with type: (b, s)
Where: ‘s’ is a rigid type variable
       ‘b’ is a rigid type variable
Relevant bindings include
  a :: a
  st'' :: s
  f :: a -> b
  st' :: s
  st :: s
  rs :: State s a
  rf :: State s (a -> b)
  (<*>) :: State s (a -> b) -> State s a -> State s b
\end{verbatim}
\end{overprint}
\end{frame}

\begin{frame}[fragile]
Am wichtigsten ist die \texttt{Monad}-Instanz:
\begin{overprint}
\onslide<2>
\begin{minted}{haskell}
instance Monad (State s) where
    return   = pure
    rs >>= f = State $ \st -> _

    
    
\end{minted}
\onslide<3>
\begin{minted}{haskell}
instance Monad (State s) where
    return   = pure
    rs >>= f = State $ \st ->
                let (a,st') = runState rs st
                in _
    
\end{minted}
\onslide<4>
\begin{minted}{haskell}
instance Monad (State s) where
    return   = pure
    rs >>= f = State $ \st ->
                let (a,st') = runState rs st
                    rs'    = f a
                in _
\end{minted}
\onslide<5>
\begin{minted}{haskell}
instance Monad (State s) where
    return   = pure
    rs >>= f = State $ \st ->
                let (a,st') = runState rs st
                    rs'    = f a
                in runState rs' st'
\end{minted}
\end{overprint}
\bigskip
\scriptsize
\begin{overprint}
\onslide<2>
\begin{verbatim}
Found hole ‘_’ with type: (b, s)
Where: ‘s’ is a rigid type variable
       ‘b’ is a rigid type variable
Relevant bindings include
  st :: s
  f :: a -> State s b
  rs :: State s a
  (>>=) :: State s a -> (a -> State s b) -> State s b
\end{verbatim}
\onslide<3>
\begin{verbatim}
Found hole ‘_’ with type: (b, s)
Where: ‘s’ is a rigid type variable
       ‘b’ is a rigid type variable
Relevant bindings include
  a :: a
  st' :: s
  st :: s
  f :: a -> State s b
  rs :: State s a
  (>>=) :: State s a -> (a -> State s b) -> State s b
\end{verbatim}
\onslide<4>
\begin{verbatim}
Found hole ‘_’ with type: (b, s)
Where: ‘s’ is a rigid type variable
       ‘b’ is a rigid type variable
Relevant bindings include
  rs' :: State s b
  a :: a
  st' :: s
  st :: s
  f :: a -> State s b
  rs :: State s a
  ...
\end{verbatim}
\end{overprint}
%$
\end{frame}

\section{Kombination von Monaden}

\subsection{Beispiel}
\begin{frame}[fragile]
Wir hatten letzte Woche die \texttt{Maybe}-Monade mit dem folgenden Anwendugsfall: \bigskip

\begin{minted}{haskell}
f = do folder <- getInbox
       mail   <- getFirstMail folder
       header <- getHeader mail
       return header
\end{minted}
\pause
Nun möchten wir aus irgendeinem Grund (Logging, Netzwerk, ..) zwischen dem \texttt{getInbox} und dem \texttt{getFirstMail} eine \texttt{IO}-Aktion ausführen.
\bigskip \pause

Problem: \texttt{IO /= Maybe}\bigskip

\pause
Als Konsequenz können wir die \texttt{do}-Notation nicht verwenden - wir fallen also wieder zurück auf die hässliche Notation:
\end{frame}

\begin{frame}[fragile]
\begin{minted}{haskell}
f :: IO (Maybe Header)
f = case getInbox of
     (Just folder) -> 
        do
          putStrLn "debug"
          case getFirstMail folder of
            (Just mail) -> 
               case getHeader mail of
                 (Just head) -> return $ return head
                 Nothing     -> return Nothing
            Nothing     -> return Nothing
     Nothing       -> return Nothing
\end{minted}
%$
\end{frame}

\begin{frame}[fragile]
Dieser Code ist ohne Frage umständlich und unschön. Stellt sich die Frage, ob wir nicht so etwas wie \texttt{MaybeIO} bauen können, sodass wir wieder \texttt{do}-Notation verwenden können.\bigskip

\pause
Also kombinieren wir es (ähnlich zur \texttt{State}-Monade):

\begin{minted}{haskell}
newtype MaybeIO a = MaybeIO { runMaybeIO :: IO (Maybe a) }
\end{minted}

\pause
Dieses liefert uns zwei Funktionen:

\begin{minted}{haskell}
MaybeIO    :: IO (Maybe a) -> MaybeIO a
runMaybeIO :: MaybeIO a -> IO (Maybe a)
\end{minted}

Also eine Funktion, um einen Wert in unsere neue Monade zu bekommen und eine Funktion um dieses wieder Rückgängig zu machen.
\end{frame}

\subsection{F/A/M}
\begin{frame}[fragile]
Fangen wir mit der \texttt{Functor}-Instanz an:
\begin{overprint}
\onslide<2>
\begin{minted}{haskell}
instance Functor MaybeIO where
  fmap f input = _

  
  
                 
\end{minted}
\onslide<3>
\begin{minted}{haskell}
instance Functor MaybeIO where
  fmap f input = _
               where
                 unwrapped = runMaybeIO input

                 
\end{minted}
\onslide<4>
\begin{minted}{haskell}
instance Functor MaybeIO where
  fmap f input = _
               where
                 unwrapped = runMaybeIO input
                 fmapped = fmap (fmap f) unwrapped

\end{minted}
\onslide<5>
\begin{minted}{haskell}
instance Functor MaybeIO where
  fmap f input = _
               where
                 unwrapped = runMaybeIO input
                 fmapped = fmap (fmap f) unwrapped
                 wrapped = MaybeIO fmapped
\end{minted}
\onslide<6-7>
\begin{minted}{haskell}
instance Functor MaybeIO where
  fmap f input = wrapped
               where
                 unwrapped = runMaybeIO input
                 fmapped = fmap (fmap f) unwrapped
                 wrapped = MaybeIO fmapped
\end{minted}
\end{overprint}
\bigskip
\scriptsize
\begin{overprint}
\onslide<2>
\begin{verbatim}
Found hole ‘_’ with type: MaybeIO b
Where: ‘b’ is a rigid type variable
Relevant bindings include
  input :: MaybeIO a
  f :: a -> b
  fmap :: (a -> b) -> MaybeIO a -> MaybeIO b
\end{verbatim}
\onslide<3>
\begin{verbatim}
Found hole ‘_’ with type: MaybeIO b
Where: ‘b’ is a rigid type variable
Relevant bindings include
  unwrapped :: IO (Maybe a)
  input :: MaybeIO a
  f :: a -> b
  fmap :: (a -> b) -> MaybeIO a -> MaybeIO b
\end{verbatim}
\onslide<4>
\begin{verbatim}
Found hole ‘_’ with type: MaybeIO b
Where: ‘b’ is a rigid type variable
Relevant bindings include
  fmapped :: IO (Maybe b)
  unwrapped :: IO (Maybe a)
  input :: MaybeIO a
  f :: a -> b
  fmap :: (a -> b) -> MaybeIO a -> MaybeIO b
\end{verbatim}
\onslide<5>
\begin{verbatim}
Found hole ‘_’ with type: MaybeIO b
Where: ‘b’ is a rigid type variable
Relevant bindings include
  wrapped :: MaybeIO b
  fmapped :: IO (Maybe b)
  unwrapped :: IO (Maybe a)
  input :: MaybeIO a
  f :: a -> b
  fmap :: (a -> b) -> MaybeIO a -> MaybeIO b
\end{verbatim}
\onslide<7>
\normalsize
oder kurz:
\begin{minted}{haskell}
instance Functor MaybeIO where
  fmap f = MaybeIO . fmap (fmap f) . runMaybeIO
\end{minted}
\end{overprint}
\end{frame}

\begin{frame}[fragile]
\texttt{Applicative}:
\begin{overprint}
\onslide<2>
\begin{minted}{haskell}
instance Applicative MaybeIO where
  pure a  = _
  f <*> x = undefined
  
                 
\end{minted}
\onslide<3>
\begin{minted}{haskell}
instance Applicative MaybeIO where
  pure a  = MaybeIO $ _
  f <*> x = undefined
  
                 
\end{minted}
\onslide<4>
\begin{minted}{haskell}
instance Applicative MaybeIO where
  pure a  = MaybeIO $ pure $ _
  f <*> x = undefined
  
                 
\end{minted}
\onslide<5>
\begin{minted}{haskell}
instance Applicative MaybeIO where
  pure a  = MaybeIO $ pure $ pure $ _
  f <*> x = undefined
  
                 
\end{minted}
\onslide<6>
\begin{minted}{haskell}
instance Applicative MaybeIO where
  pure a  = MaybeIO $ pure $ pure $ a
  f <*> x = undefined
  
                 
\end{minted}
\onslide<7>
\begin{minted}{haskell}
instance Applicative MaybeIO where
  pure a  = MaybeIO . pure . pure $ a
  f <*> x = undefined
  
                 
\end{minted}
\onslide<8>
\begin{minted}{haskell}
instance Applicative MaybeIO where
  pure    = MaybeIO . pure . pure
  f <*> x = _
  
                 
\end{minted}
\onslide<9>
\begin{minted}{haskell}
instance Applicative MaybeIO where
  pure    = MaybeIO . pure . pure
  f <*> x = MaybeIO $ _
  
                 
\end{minted}
\onslide<10>
\begin{minted}{haskell}
instance Applicative MaybeIO where
  pure    = MaybeIO . pure . pure
  f <*> x = MaybeIO $ _
            where
             f' = runMaybeIO f
             x' = runMaybeIO x
\end{minted}
\onslide<11>
\begin{minted}{haskell}
instance Applicative MaybeIO where
  pure    = MaybeIO . pure . pure
  f <*> x = MaybeIO $ (<*>) <$> f' <*> x'
            where
             f' = runMaybeIO f
             x' = runMaybeIO x
\end{minted}
\end{overprint}
\bigskip
\scriptsize
\begin{overprint}
\onslide<2>
\begin{verbatim}
Found hole ‘_’ with type: MaybeIO a
Where: ‘a’ is a rigid type variable
Relevant bindings include
  a :: a
  pure :: a -> MaybeIO a
\end{verbatim}
\onslide<3>
\begin{verbatim}
Found hole ‘_’ with type: IO (Maybe a)
Where: ‘a’ is a rigid type variable
Relevant bindings include
  a :: a
  pure :: a -> MaybeIO a
\end{verbatim}
\onslide<4>
\begin{verbatim}
Found hole ‘_’ with type: Maybe a
Where: ‘a’ is a rigid type variable
Relevant bindings include
  a :: a
  pure :: a -> MaybeIO a
\end{verbatim}
\onslide<5>
\begin{verbatim}
Found hole ‘_’ with type: a
Where: ‘a’ is a rigid type variable
Relevant bindings include
  a :: a
  pure :: a -> MaybeIO a
\end{verbatim}
\onslide<8>
\begin{verbatim}
Found hole ‘_’ with type: MaybeIO b
Where: ‘b’ is a rigid type variable
Relevant bindings include
  x :: MaybeIO a
  f :: MaybeIO (a -> b)
  (<*>) :: MaybeIO (a -> b) -> MaybeIO a -> MaybeIO b
\end{verbatim}
\onslide<9>
\begin{verbatim}
Found hole ‘_’ with type: IO (Maybe b)
Where: ‘b’ is a rigid type variable
Relevant bindings include
  x :: MaybeIO a
  f :: MaybeIO (a -> b)
  (<*>) :: MaybeIO (a -> b) -> MaybeIO a -> MaybeIO b
\end{verbatim}
\onslide<10>
\begin{verbatim}
Found hole ‘_’ with type: IO (Maybe b)
Where: ‘b’ is a rigid type variable
Relevant bindings include
  f' :: IO (Maybe (a -> b))
  x' :: IO (Maybe a)
  x :: MaybeIO a
  f :: MaybeIO (a -> b)
  (<*>) :: MaybeIO (a -> b) -> MaybeIO a -> MaybeIO b
\end{verbatim}
\onslide<11>
\normalsize
Das erste \texttt{(<*>)} ist \texttt{Applicative} auf \texttt{Maybe} und es wird in \texttt{Applicative} \texttt{<*>} von \texttt{IO} hineingemappt.
\end{overprint}
\end{frame}

\begin{frame}[fragile]
\texttt{Monad}:
\begin{overprint}
\onslide<2>
\begin{minted}{haskell}
instance Monad MaybeIO where
  return  = pure
  x >>= f = MaybeIO $ _
\end{minted}
\onslide<3>
\begin{minted}{haskell}
instance Monad MaybeIO where
  return  = pure
  x >>= f = MaybeIO $ _
            where
              x'  = runMaybeIO x
\end{minted}
\onslide<4>
\begin{minted}{haskell}
instance Monad MaybeIO where
  return  = pure
  x >>= f = MaybeIO $ x' >>= _
            where
              x'  = runMaybeIO x
\end{minted}
\onslide<5>
\begin{minted}{haskell}
instance Monad MaybeIO where
  return  = pure
  x >>= f = MaybeIO $ x' >>= _ . fmap f
            where
              x'  = runMaybeIO x
\end{minted}
\onslide<6>
\begin{minted}{haskell}
instance Monad MaybeIO where
  return  = pure
  x >>= f = MaybeIO $ x' >>= runMaybeIO . _ . fmap f
            where
              x'  = runMaybeIO x
\end{minted}
\onslide<7>
\begin{minted}{haskell}
instance Monad MaybeIO where
  return  = pure
  x >>= f = MaybeIO $ x' >>= runMaybeIO . mb . fmap f
            where
              x'  = runMaybeIO x
              mb :: Maybe (MaybeIO a) -> MaybeIO a
              mb a = _
\end{minted}
\onslide<8>
\begin{minted}{haskell}
instance Monad MaybeIO where
  return  = pure
  x >>= f = MaybeIO $ x' >>= runMaybeIO . mb . fmap f
            where
              x'  = runMaybeIO x
              mb :: Maybe (MaybeIO a) -> MaybeIO a
              mb (Just a) = _
              mb Nothing  = undefined
\end{minted}
\onslide<9>
\begin{minted}{haskell}
instance Monad MaybeIO where
  return  = pure
  x >>= f = MaybeIO $ x' >>= runMaybeIO . mb . fmap f
            where
              x'  = runMaybeIO x
              mb :: Maybe (MaybeIO a) -> MaybeIO a
              mb (Just a) = a
              mb Nothing  = _
\end{minted}
\onslide<10>
\begin{minted}{haskell}
instance Monad MaybeIO where
  return  = pure
  x >>= f = MaybeIO $ x' >>= runMaybeIO . mb . fmap f
            where
              x'  = runMaybeIO x
              mb :: Maybe (MaybeIO a) -> MaybeIO a
              mb (Just a) = a
              mb Nothing  = MaybeIO $ _
\end{minted}
\onslide<11>
\begin{minted}{haskell}
instance Monad MaybeIO where
  return  = pure
  x >>= f = MaybeIO $ x' >>= runMaybeIO . mb . fmap f
            where
              x'  = runMaybeIO x
              mb :: Maybe (MaybeIO a) -> MaybeIO a
              mb (Just a) = a
              mb Nothing  = MaybeIO $ return _
\end{minted}
\onslide<12>
\begin{minted}{haskell}
instance Monad MaybeIO where
  return  = pure
  x >>= f = MaybeIO $ x' >>= runMaybeIO . mb . fmap f
            where
              x'  = runMaybeIO x
              mb :: Maybe (MaybeIO a) -> MaybeIO a
              mb (Just a) = a
              mb Nothing  = MaybeIO $ return Nothing
\end{minted}
\end{overprint}
\bigskip
\scriptsize
\begin{overprint}
\onslide<2>
\begin{verbatim}
Found hole ‘_’ with type: IO (Maybe b)
Where: ‘b’ is a rigid type variable
Relevant bindings include
  f :: a -> MaybeIO b
  x :: MaybeIO a
  (>>=) :: MaybeIO a -> (a -> MaybeIO b) -> MaybeIO b
\end{verbatim}
\onslide<3>
\begin{verbatim}
Found hole ‘_’ with type: IO (Maybe b)
Where: ‘b’ is a rigid type variable
Relevant bindings include
  x' :: IO (Maybe a)
  f :: a -> MaybeIO b
  x :: MaybeIO a
  (>>=) :: MaybeIO a -> (a -> MaybeIO b) -> MaybeIO b
\end{verbatim}
\onslide<4>
\begin{verbatim}
Found hole ‘_’ with type: Maybe a -> IO (Maybe b)
Where: ‘a’ is a rigid type variable
       ‘b’ is a rigid type variable
Relevant bindings include
  x' :: IO (Maybe a)
  f :: a -> MaybeIO b
  x :: MaybeIO a
  (>>=) :: MaybeIO a -> (a -> MaybeIO b) -> MaybeIO b
\end{verbatim}
\onslide<5>
\begin{verbatim}
Found hole ‘_’ with type: Maybe (MaybeIO b) -> IO (Maybe b)
Where: ‘b’ is a rigid type variable
Relevant bindings include
  x' :: IO (Maybe a)
  f :: a -> MaybeIO b
  x :: MaybeIO a
  (>>=) :: MaybeIO a -> (a -> MaybeIO b) -> MaybeIO b
\end{verbatim}
\onslide<6>
\begin{verbatim}
Found hole ‘_’ with type: Maybe (MaybeIO b) -> MaybeIO b
Where: ‘b’ is a rigid type variable
Relevant bindings include
  x' :: IO (Maybe a)
  f :: a -> MaybeIO b
  x :: MaybeIO a
  (>>=) :: MaybeIO a -> (a -> MaybeIO b) -> MaybeIO b
\end{verbatim}
\onslide<7>
\begin{verbatim}
Found hole ‘_’ with type: MaybeIO a1
Where: ‘a1’ is a rigid type variable
Relevant bindings include
  a :: Maybe (MaybeIO a1)
  mb :: Maybe (MaybeIO a1) -> MaybeIO a1
  f :: a -> MaybeIO b
  x :: MaybeIO a
  (>>=) :: MaybeIO a -> (a -> MaybeIO b) -> MaybeIO b
\end{verbatim}
\onslide<8>
\begin{verbatim}
Found hole ‘_’ with type: MaybeIO a1
Where: ‘a1’ is a rigid type variable
Relevant bindings include
  a :: MaybeIO a1
  mb :: Maybe (MaybeIO a1) -> MaybeIO a1
  f :: a -> MaybeIO b
  x :: MaybeIO a
  (>>=) :: MaybeIO a -> (a -> MaybeIO b) -> MaybeIO b
\end{verbatim}
\onslide<9>
\begin{verbatim}
Found hole ‘_’ with type: MaybeIO a1
Where: ‘a1’ is a rigid type variable
Relevant bindings include
  mb :: Maybe (MaybeIO a1) -> MaybeIO a1
  f :: a -> MaybeIO b
  x :: MaybeIO a
  (>>=) :: MaybeIO a -> (a -> MaybeIO b) -> MaybeIO b
\end{verbatim}
\onslide<10>
\begin{verbatim}
Found hole ‘_’ with type: IO (Maybe a1)
Where: ‘a1’ is a rigid type variable
Relevant bindings include
  mb :: Maybe (MaybeIO a1) -> MaybeIO a1
  f :: a -> MaybeIO b
  x :: MaybeIO a
  (>>=) :: MaybeIO a -> (a -> MaybeIO b) -> MaybeIO b
\end{verbatim}
\onslide<11>
\begin{verbatim}
Found hole ‘_’ with type: Maybe a1
Where: ‘a1’ is a rigid type variable
Relevant bindings include
  mb :: Maybe (MaybeIO a1) -> MaybeIO a1
  f :: a -> MaybeIO b
  x :: MaybeIO a
  (>>=) :: MaybeIO a -> (a -> MaybeIO b) -> MaybeIO b
\end{verbatim}
\end{overprint}
\end{frame}

\subsection{Beispiel revisited}
\begin{frame}[fragile]
Da wir nun eine Monade definiert haben, können wir ja wieder \texttt{do} nutzen:
\begin{minted}{haskell}
f = do i <- getInbox
       putStrLn "debug"
       m <- getFirstMail i
       h <- getHeader m
       return h
\end{minted}
\end{frame}
\begin{frame}[fragile]
Allerdings:
\begin{minted}{text}
    Couldn't match type Maybe with MaybeIO
    Expected type: MaybeIO Inbox
      Actual type: Maybe Inbox
    In a stmt of a 'do' block: in <- getInbox

    Couldn't match type IO with MaybeIO
    Expected type: MaybeIO ()
      Actual type: IO ()
    In a stmt of a 'do' block: putStrLn "debug"

    Couldn't match type Maybe with MaybeIO
    Expected type: MaybeIO Mail
      Actual type: Maybe Mail
    In a stmt of a 'do' block: m <- getFirstMail i
    
    Couldn't match type Maybe with MaybeIO
    Expected type: MaybeIO Header
      Actual type: Maybe Header
    In a stmt of a 'do' block: h <- getHeader m
\end{minted}
\end{frame}

\begin{frame}[fragile]
Wir brauchen also zwei Konverter:\smallskip

\begin{itemize}
 \item \texttt{Maybe -> MaybeIO}
 \item \texttt{IO -> MaybeIO}
\end{itemize}
\smallskip

\pause
Aber wir haben schon alles, was wir brauchen, wenn wir uns nur Folgendes klar machen:\smallskip

\begin{minted}{haskell}
return  :: Maybe a -> IO (Maybe a) -- return von IO
MaybeIO :: IO (Maybe a) -> MaybeIO a
\end{minted}
\smallskip

\pause
und
\begin{minted}{haskell}
Just      :: a -> Maybe a
fmap Just :: IO a -> IO (Maybe a)
\end{minted}
\end{frame}

\begin{frame}[fragile]
Somit wird unser Code von oben: \smallskip

\begin{minted}{haskell}
f = do i <- MaybeIO (return (getInbox))
       MaybeIO (fmap Just (putStrLn "debug"))
       m <- MaybeIO (return (getFirstMail i))
       h <- MaybeIO (return (getHeader m))
       return h
\end{minted}
\pause
\bigskip

Zwar können wir nun \texttt{do} nutzen, aber das sieht doch eher hässlich aus. Außerdem ist so viel Code doppelt!
\end{frame}

\subsection{Finale Version}

\begin{frame}[fragile]
Wenn wir Muster finden, dann lagern wir sie doch einfach in Funktionen aus!\bigskip

\begin{minted}{haskell}
liftMaybe :: Maybe a -> MaybeIO a
liftMaybe x = MaybeIO (return x)

liftIO :: IO a -> MaybeIO a
liftIO x = MaybeIO (fmap Just x)
\end{minted}
\pause
\smallskip

und wir erhalten:\smallskip

\begin{minted}{haskell}
f = do i <- liftMaybe getInbox
       liftIO $ putStrLn "debug"
       m <- liftMaybe $ getFirstMail i
       h <- liftMaybe $ getHeader m
       return h
\end{minted}
%$
\end{frame}

\section{Monad-Transformer}

\subsection{Recap}

\begin{frame}[fragile]
Wenn wir uns nochmals ansehen, welche Eigenschaft der \texttt{IO}-Monade wir genutzt haben, dann fällt uns auf:
\pause
\begin{minted}{haskell}
instance Functor MaybeIO where
  fmap f = MaybeIO . fmap (fmap f) . runMaybeIO
\end{minted}
\texttt{fmap} von \texttt{IO} als \texttt{Functor}
\pause
\begin{minted}{haskell}
instance Applicative MaybeIO where
  pure    = MaybeIO . pure . pure
  f <*> x = MaybeIO $ (<*>) <$> (runMaybeIO f)
                            <*> (runMaybeIO x)
\end{minted}
\texttt{pure} und \texttt{<*>} von \texttt{IO} als \texttt{Applicative}
\pause
\begin{minted}{haskell}
instance Monad MaybeIO where
  return = pure
  x >>= f = MaybeIO $ (runMaybeIO x)
                      >>= runMaybeIO . mb . fmap f
            where
              mb (Just a) = a
              mb Nothing = MaybeIO $ return Nothing
\end{minted}
\texttt{return} und \texttt{>>=} von \texttt{IO} als \texttt{Monad}
\end{frame}

\begin{frame}[fragile]
Uns fällt auf: Wir verwenden gar keine intrisischen Eigenschaften von \texttt{IO}.\bigskip

Also können wir \texttt{IO} auch durch jede andere Monade ersetzten. Dies nennt man dann \emph{Monad Transformer}.\bigskip

\begin{minted}{haskell}
newtype MaybeT m a = MaybeT { runMaybeT :: m (Maybe a) }
\end{minted}
\end{frame}

\begin{frame}[fragile]
\begin{overprint}
\onslide<1>
Und der Code von eben...
\onslide<2>
...wird zu:
\end{overprint}
\begin{overprint}
\onslide<1>
\begin{minted}{haskell}
instance Functor MaybeIO where
  fmap f = MaybeIO . fmap (fmap f) . runMaybeIO

instance Applicative MaybeIO where
  pure    = MaybeIO . pure . pure
  f <*> x = MaybeIO $ (<*>) <$> (runMaybeIO f)
                            <*> (runMaybeIO x)

instance Monad MaybeIO where
  return = pure
  x >>= f = MaybeIO $ (runMaybeIO x)
                      >>= runMaybeIO . mb . fmap f
            where
              mb (Just a) = a
              mb Nothing = MaybeIO $ return Nothing
\end{minted}
\onslide<2>
\begin{minted}{haskell}
instance Functor m => Functor (MaybeT m) where
  fmap f = MaybeT  . fmap (fmap f) . runMaybeT

instance Applicative m => Applicative (MaybeT m) where
  pure    = MaybeT  . pure . pure
  f <*> x = MaybeT  $ (<*>) <$> (runMaybeT  f)
                            <*> (runMaybeT  x)

instance Monad m => Monad (MaybeT m) where
  return = pure
  x >>= f = MaybeT  $ (runMaybeT  x)
                      >>= runMaybeT  . mb . fmap f
            where
              mb (Just a) = a
              mb Nothing = MaybeT  $ return Nothing
\end{minted}
\end{overprint}
\end{frame}

\begin{frame}[fragile]
Frage: Wie realisieren wir nun \texttt{liftIO} etc.?\bigskip

\pause
Über Typklassen! Dafür sind sie schließlich da! \smallskip
\begin{minted}{haskell}
class Monad m => MonadIO m where
    liftIO :: IO a -> m a
\end{minted}
Wir verlangen einfach, dass \texttt{IO} irgendwie verarbeitet werden muss.\bigskip

\pause
Genereller:\smallskip

\begin{minted}{haskell}
class MonadTrans t where
    lift :: (Monad m) => m a -> t m a
\end{minted}
\pause
\smallskip

Dies ist die allgemeine Form für \emph{additive} Monaden. Mit \texttt{lift} heben wir uns eine monadische Ebene höher.
\pause
\begin{important}
Wichtig: \texttt{IO} ist nicht additiv! Es gibt keinen \texttt{IO-T}!
\end{important}
\end{frame}

\subsection{Beispiele}

\begin{frame}[fragile]
Wir haben schon ein paar Monaden kennengelernt. Diese sind \emph{fast} alle additiv. Wir können somit folgendes bauen:\bigskip

\pause
\begin{minted}{haskell}
data MyMonadStack a = StateT MyState 
                            (EitherT String 
                                     (MaybeT (IO a)))
\end{minted}
\pause
Wie schreiben wir nun Code dafür?
\begin{minted}{haskell}
bsp :: MyMonadStack ()
bsp = do
    a <- fun                
    -- fun  :: StateT Mystate (EitherT String (MaybeT (IO Int)))
    b <- lift $ fun2        
    -- fun2 :: EitherT String (MaybeT (IO Int))
    c <- lift . lift $ fun3 
    -- fun3 :: MaybeT (IO Int))
    liftIO $ putStrLn "foo"
    -- putStrLn :: IO ()
\end{minted}
%$
\end{frame}

\begin{frame}[fragile]
Weitere Monaden, die hierbei häufig vorkommen, sind z.B.
\begin{description}
 \item[\texttt{ReaderT}] für ein read-only-Environment (z.B. Konfiguration)
 \pause
 \item[\texttt{WriterT}] für ein write-only-Environment (z.B. für Logging)
 \pause
 \item[\texttt{StateT}] für einen globalen State
 \pause
 \item[\texttt{EitherT}] für fehlschlagbare Operationen (mit Fehlermeldung)
 \pause
 \item[\texttt{MaybeT}] für fehlschlagbare Operationen (ohne Fehlermeldung)
\end{description}
\pause
\bigskip

Je nachdem, welche Möglichkeiten man haben möchte, kann man diese miteinander kombinieren.
\end{frame}

\begin{frame}[fragile]
Auch kommt es auf die Reihenfolge an:
\begin{minted}{haskell}
StateT MyState (EitherT String (Identity a))
\end{minted}
kann fehlschlagen, aber man kommt nach dem Fehlschlag noch an den \texttt{State} dran,\pause wohingegen
\begin{minted}{haskell}
EitherT String (StateT MyState (Identity a))
\end{minted}
nur die Fehlermeldung liefert und den State schon entsorgt hat.\\\par
\pause
Häufig findet man einen Read-Write-State-Transformer, kurz \texttt{RWST}.\\
\pause
Echtweltprogramme sind oft durch einen \texttt{RWST IO} mit der Außenwelt verbunden.
\end{frame}

\begin{frame}[fragile]
Ein Echtwelt-Beispiel könnte etwa der folgende Aufruf sein:\bigskip

\begin{minted}{haskell}
data Env = Env { filename :: String }

readInputs :: ReaderT Env IO String
readInputs = do
           e <- ask
           f <- liftIO $ readFile (filename e)
           return f
\end{minted}
\pause
Dieser Aufruf liest einen Dateinamen aus einem Environment, kann per \texttt{liftIO} \texttt{IO}-Aktionen ausführen und das Ergebnis (den String mit dem Dateiinhalt) zurückliefern.
\end{frame}

\begin{frame}[fragile]
Noch ein Beispiel aus einem Spiel könnte sein:\smallskip

\begin{minted}{haskell}
mainLoop :: RWST Env () State IO ()
mainLoop = do
         e <- ask
         f <- liftIO $ getUserInput (keySettings e)
         oldWorld <- get
         newWorld <- updateWorld f oldWorld
         put newWorld
         unless (f == endKey e) mainLoop
\end{minted}
\bigskip

\pause
Dies ist eine klassische Game-Loop, bestehend aus Konfigurationen im \texttt{Env} (Key settings), \texttt{IO} (User-Input abfragen), Update des internen Zustands (\texttt{updateWorld}) und das schreiben des neuen Zustandes (\texttt{put newWorld}).\smallskip

\pause
Wichtig: \texttt{updateWorld} ist pure!
\end{frame}

\end{document}